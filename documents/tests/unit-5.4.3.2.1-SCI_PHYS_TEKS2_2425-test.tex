\documentclass[]{exam}
\usepackage{marvosym}

%...TikZ & PGF
\usepackage{pgfplots}
\pgfplotsset{compat=1.11}
\tikzset{>=latex}
\usetikzlibrary{calc,math}
\usepackage{tikzsymbols}
\usepgfplotslibrary{fillbetween}
\usetikzlibrary{decorations.markings} 
\usetikzlibrary{arrows.meta} %...APP2 for arrows as objects and images
\usetikzlibrary{backgrounds} %...For shading portions of graphs
\usetikzlibrary{patterns} %...Unit 5 Problems
\usetikzlibrary{shapes.geometric} %...For drawing cylinders in Unit 2
\tikzset{
    mark position/.style args={#1(#2)}{
        postaction={
            decorate,
            decoration={
                markings,
                mark=at position #1 with \coordinate (#2);
            }
        }
    }
} %...See https://tex.stackexchange.com/questions/43960/define-node-at-relative-coordinates-of-draw-plot

\tikzset{
    declare function = {trajectoryequation10(\x,\vi,\thetai)= tan(\thetai)*\x - 10*\x^2/(2*(\vi*cos(\thetai))^2);},
    declare function = {trajectoryequation(\x,\vi,\thetai)= tan(\thetai)*\x - 9.8*\x^2/(2*(\vi*cos(\thetai))^2);},
    declare function = {patheq(\x,\yi,\vi,\thetai)= \yi + tan(\thetai)*\x - 9.8*\x^2/(2*(\vi*cos(\thetai))^2);},
    declare function = {patheqten(\x,\yi,\vi,\thetai)= \yi + tan(\thetai)*\x - 10*\x^2/(2*(\vi*cos(\thetai))^2);} %like patheq but with gravity = 10
}

%...siunitx
\usepackage{siunitx}
\DeclareSIUnit{\nothing}{\relax}
\def\mymu{\SI{}{\micro\nothing} }
\DeclareSIUnit\mmHg{mmHg}
\DeclareSIUnit{\mile}{mi}
%...NOTE: "The product symbol between the number and unit is set using the quantity-product option."

%...Other
\usepackage{amsthm}
\usepackage{amsmath}
\usepackage{amssymb}
\usepackage{cancel}
\usepackage{subcaption}
\usepackage{dashrule}
\usepackage{enumitem}
\usepackage{fontawesome}
\usepackage{multicol}
\usepackage{glossaries}
%\numberwithin{equation}{section}
\numberwithin{figure}{section}
\usepackage{float}
\usepackage{twemojis} %...twitter emojis
\usepackage{utfsym}
\newcommand{\R}{\mathbb{R}} %...real number symbol
\usepackage{graphicx}
\graphicspath{ {../Figures/} }
\usepackage{hyperref}
\hypersetup{colorlinks=true,
    linkcolor=blue,
    filecolor=magenta,
    urlcolor=cyan,}
\urlstyle{same}
\newcommand{\hdashline}{{\hdashrule{\textwidth}{0.5pt}{0.8mm}}}
\newcommand{\hgraydashline}{{\color{lightgray} \hdashrule{0.99\textwidth}{1pt}{0.8mm}}}

%...Miscellaneous user-defined symbols
\newcommand{\fnet}{F_{\text{net}}} %...For net force
\newcommand{\bvec}[1]{\vec{\mathbf{#1}}} %...bold vector
\newcommand{\bhat}[1]{\,\hat{\mathbf{#1}}} %...bold hat vector
\newcommand{\que}{\mathord{?}}  %...Question mark symbol in equation env
%...Define thick horizontal rule for examples:
\newcommand{\hhrule}{\hrule\hrule}
\let\oldtexttt\texttt% Store \texttt
\renewcommand{\texttt}[2][black]{\textcolor{#1}{\ttfamily #2}}% 

%...For use in the exam document class
\newif\ifprintmetasolutions


%...Decreases space above and below align and gather enironment
\makeatletter
\g@addto@macro\normalsize{%
  \setlength\abovedisplayskip{-3pt}
  \setlength\belowdisplayskip{6pt} 
}
\makeatother





\usepackage[margin=1in]{geometry}
\usepackage[figurewithin=none]{caption}
\usepackage{exam-randomizechoices}

\CorrectChoiceEmphasis{\color{red}\bfseries}
\renewcommand{\solutiontitle}{\noindent\textbf{\textcolor{red}{Solution:}}\enspace}

\usepackage{OutilsGeomTikz}
\usepackage{utfsym} %...Symbols in Unit 7 Problems
\usepackage{tabu} %...Symbols in Unit 7 Problems

%...For use in Unit 2            %    
\setlength{\columnsep}{2cm}      %
\setlength{\columnseprule}{1pt}  %
\usepackage[none]{hyphenat}      %
%%%%%%%%%%%%%%%%%%%%%%%%%%%%%%%%%

%...For use in Unit 11 on Waves:
\pgfdeclarehorizontalshading{visiblelight}{50bp}{  %
color(0.00000000000000bp)=(red);                   %
color(8.33333333333333bp)=(orange);                %
color(16.66666666666670bp)=(yellow);               %
color(25.00000000000000bp)=(green);                %
color(33.33333333333330bp)=(cyan);                 %
color(41.66666666666670bp)=(blue);                 %
color(50.00000000000000bp)=(violet)                %
}                                                  %

\newcommand{\checkbox}[1]{%
  \ifnum#1=1
    \makebox[0pt][l]{\raisebox{0.15ex}{\hspace{0.1em}\Large$\checkmark$}}%
  \fi
  $\square$%
}
%%%%%%%%%%%%%%%%%%%%%%%%%%%%%%%%%%%%%%%%%%%%%%%%%%%%

%...If using circuitikz package:
% \ctikzset{bipoles/battery1/height=0.5}
% \ctikzset{bipoles/battery1/width=0.25}
% \ctikzset{bipoles/resistor/height=0.15}
% \ctikzset{bipoles/resistor/width=0.4}
\usepackage{mdframed}

\setrandomizerseed{1}


\firstpageheader{}{\large\textbf{DO NOT WRITE ON THIS DOCUMENT}}{}
\runningheader{}{}{}


\begin{document}
\begin{questions}
\question 
After a satellite is launched into frictionless space, the amount of force needed to keep it moving at a constant speed equals

\begin{randomizechoices}
    \choice one half the force with which it was fired.	
    \choice the same amount of force with which it was fired.
    \choice twice the force with which it was fired.	
    \correctchoice zero, no force is necessary to keep it moving.
\end{randomizechoices}

\question
For each diagram below: The force, $F$, equals 100 Newtons, the mass of the box is 20 kilograms, and all surfaces are frictionless. In which situation will the box experience the greatest acceleration?

\begin{center}
\begin{tikzpicture}
    \draw (0,0) -- (3,0) node[below,pos=0.5] {Diagram A};
    \begin{scope}[shift={(1,0)}]
        \draw[fill=black!10] (0,0) rectangle (1,0.7);
        \draw[thick,->] (1,0.7) -- ++(1,0.6) node[above] {$F$};        
    \end{scope}
\end{tikzpicture}
\hspace{1cm}
\begin{tikzpicture}
    \draw (0,0) -- (3,0) node[below,pos=0.5] {Diagram B};
    \begin{scope}[shift={(1,0)}]
        \draw[fill=black!10] (0,0) rectangle (1,0.7);
        \draw[thick,->] (1,0.35) -- ++(1.17,0) node[above] {$F$};        
    \end{scope}
\end{tikzpicture}
\hspace{1cm}
\begin{tikzpicture}
    \draw (0,0) -- (3,0) node[below,pos=0.5] {Diagram C};
    \begin{scope}[shift={(1,0)}]
        \draw[fill=black!10] (0,0) rectangle (1,0.7);
        \draw[thick,<-] (1,0.7)node[above=2pt] {$F$} -- ++(1,0.6);        
    \end{scope}
\end{tikzpicture}
\end{center}

\begin{randomizeoneparchoices}[norandomize]
    \choice Diagram A
    \correctchoice Diagram B
    \choice Diagram C
    \choice Not enough information given
\end{randomizeoneparchoices}

\question
The bar chart below shows the kinetic energy of a 3-kg object at two different points in time.

\begin{center}
\begin{tikzpicture}
\begin{axis}[width=7cm,height=6cm,
    % axis lines=left,
    ybar,
    symbolic x coords={initial, final},
    xtick=data,
    ylabel={Kinetic Energy (J)},
    ymin=0, ymax=10,
    ytick={0,1,...,10},
    bar width=35pt,
    enlarge x limits=0.5,
    ymajorgrids=true,
    tick pos=left,
    title={Change in Kinetic Energy}
]
\addplot[black,fill=black!20] coordinates {(initial,8) (final,2)};
\end{axis}
\end{tikzpicture}
\end{center}

Based on the data, calculate the work that was done on the object during this time period.

\begin{randomizechoices}
    \choice $+\SI{18}{J}$
    \choice $+\SI{6}{J}$
    \correctchoice $-\SI{6}{J}$
    \choice $-\SI{18}{J}$
\end{randomizechoices}

\question
The data represented in the velocity vs. time graph below is from a car traveling along a flat straight road.

\begin{center}
\begin{tikzpicture}
    \begin{axis}[width=8cm,height=5cm,
        axis lines=left,
        ymin=0, ymax=5,
        xmin=0, xmax=6,
        ylabel = {Velocity (m/s)},
        xlabel = {Time (s)},
        grid = both,
        xtick={0,1,...,10},
        ytick={0,1,...,6},
        minor x tick num=1,
    ]
    \addplot[ultra thick,domain=0:5,mark=*,samples=6] plot {3};
\end{axis}
\end{tikzpicture}
\end{center}

What is the total displacement of the car between 0 second and 5 seconds?

\begin{randomizechoices}
    \choice \SI{0}{m}
    \choice \SI{3}{m}
    \choice \SI{5}{m}
    \correctchoice \SI{15}{m}
\end{randomizechoices}

\question
A child riding a bicycle at \SI{15}{m/s} accelerates at \SI{3.0}{m/s^2} for \SI{4.0}{s}. What is the child's speed at the end of this \SI{4.0}{s} interval?

\begin{randomizechoices}
    \choice \SI{12}{m/s}
    \correctchoice \SI{27}{m/s}
    \choice \SI{3.0}{m/s}
    \choice \SI{7.0}{m/s}
\end{randomizechoices}

\question
The velocity vs. time data table below represents the motion of a \SI{2}{kg} object.

\begin{center}
    \begin{tabular}{|c|c|}
        \hline
         \textbf{Time} (s) & \textbf{Velocity} (m/s) \\ \hline
         0 & 0 \\ \hline
         1 & 1 \\ \hline
         2 & 2 \\ \hline
         3 & 3 \\ \hline
         4 & 4 \\ \hline
         5 & 5 \\ \hline
         6 & 6 \\ \hline
    \end{tabular}
\end{center}

Determine the force acting on the object from 2 to 4 seconds.

\begin{randomizechoices}
    \correctchoice \SI{2}{N}
    \choice \SI{6}{N}
    \choice \SI{3}{N}
    \choice \SI{1}{N}
\end{randomizechoices}

\question
The chart below is a record of the instantaneous velocities at the given times for a toy car used in a laboratory investigation.

\begin{center}
    \begin{tabular}{|c|c|}
        \hline
         \textbf{Time} (s) & \textbf{Velocity} (m/s) \\ \hline
         0 & $+3$ \\ \hline
         1 & $+1$ \\ \hline
         2 & $-1$ \\ \hline
         3 & $-3$ \\ \hline
    \end{tabular}
\end{center}

Which of the following statements is supported by the data?

\begin{randomizechoices}
    \choice From 0 seconds to 1 seconds, the car is speeding up.
    \choice From 1 second to 2 seconds, the car has a constant velocity.
    \choice From 2 seconds to 3 seconds, the car is slowing down.
    \correctchoice From 0 seconds to 3 seconds, the velocity is changing at a constant rate.
\end{randomizechoices}

\clearpage

\question
The data represented in the position vs. time graph below was collected from a cart moving along a horizontal track.

\begin{center}
\begin{tikzpicture}
    \begin{axis}[height=5cm,width=7cm,
        axis lines=left,
        ymin=0, ymax=80,
        xmin=0, xmax=6,
        ylabel = {Position (m)},
        xlabel = {Time (s)},
        grid = both,
        ytick={0,20,...,80},
        xtick={0,1,...,6},
        clip=false,
    ]
    \addplot[very thick] coordinates {(0,40)(1,40)(2,60)(4,60)(5,0)}; 
\end{axis}
\end{tikzpicture}
\end{center}

During which time interval is the cart moving with the greatest speed?

\begin{randomizechoices}
    \choice Between 0 to 1 seconds	
    \choice Between 1 to 2 seconds
    \choice Between 2 to 4 seconds
    \correctchoice Between 4 to 5 seconds
\end{randomizechoices}

\question
An air bag is used to safely decrease the momentum of a driver in a car accident. The air bag reduces the magnitude of the force acting on the driver by

\begin{randomizechoices}	
    \choice decreasing the distance over which the force acts on the driver.	
    \choice increasing the rate of acceleration of the driver.	
    \choice decreasing the mass of the driver.	
    \correctchoice increasing the amount of time the force acts on the driver.
\end{randomizechoices}

\question
The object illustrated below is moving horizontally. 

\begin{center}
\begin{tikzpicture}
    \draw (0,0) -- (5,0);
    \draw[thick,fill=black!10] (1.5,0) rectangle ++(1,1);
    \fill (2,0.5) circle (2pt);
    \draw[->] (2,0.5) -- ++(0,+1.3) node[above] {$F_\mathrm{N}$};
    \draw[->] (2,0.5) -- ++(0,-1.3) node[below] {$F_g$};
    \draw[->] (2,0.5) -- ++(+1.5,0) node[above,pos=0.9] {$+$} node[right] {$F_\mathrm{A}$};
    \draw[->] (2,0.5) -- ++(-1.5,0) node[left] {$F_f$};
\end{tikzpicture}
\end{center}

The magnitudes of the applied force $F_\mathrm{A}$ and the frictional force $F_f$ are equal. The object must be moving with 

\begin{randomizechoices}
    \correctchoice constant, positive velocity.
    \choice constant, negative velocity.
    \choice decreasing velocity.
    \choice increasing velocity.
\end{randomizechoices}

\clearpage

\question
The velocity vs. time graph below represents the motion of a 4 kg cart along a straight line.

\begin{center}
\begin{tikzpicture}
    \begin{axis}[width=7cm,height=5cm,
        axis lines=left,
        ymin=0, ymax=7,
        xmin=0, xmax=4,
        ylabel = {Velocity (m/s)},
        xlabel = {Time (s)},
        grid = both,
        ytick={0,1,...,7},
        xtick={0,1,...,4},
        minor x tick num=1,
    ]
    \addplot[ultra thick] coordinates {(0,6)(1,6)(1.25,3)(3.5,3)};
\end{axis}
\end{tikzpicture}
\end{center}

What is the change in momentum of the \SI{4}{kg} cart between $t = 0$ and $t = 3$ seconds?

\begin{randomizechoices}
    \choice $\SI{-3}{kg\cdot m/s}$
    \choice $\SI{0}{kg\cdot m/s}$
    \correctchoice $\SI{-12}{kg\cdot m/s}$
    \choice $\SI{-6}{kg\cdot m/s}$
\end{randomizechoices}

\question
The graph below shows the force exerted on a box sliding across the floor with a positive velocity.

\begin{center}
\begin{tikzpicture}
    \begin{axis}[width=7cm,height=5cm,
        axis lines=left,
        ymin=0, ymax=120,
        xmin=0, xmax=30,
        ylabel = {Force (N)},
        xlabel = {Displacement (m)},
        grid = both,
        ytick={0,20,...,120},
        xtick={0,5,...,30},
        grid=both
    ]
    \addplot[ultra thick] coordinates {(0,100)(20,100)(30,0)};
\end{axis}
\end{tikzpicture}
\end{center}

What is the change in kinetic energy of the box as it slides through a displacement of \SI{20}{m} with a \SI{100}{N} net force acting on it?

\begin{randomizechoices}
    \choice \SI{0}{J}
    \choice \SI{20}{J}
    \choice \SI{100}{J}
    \correctchoice \SI{2000}{J}
\end{randomizechoices}


\clearpage

\question
Two carts collide and bounce off one another. The diagram below shows their motion after the collision.


\begin{center}
\begin{tikzpicture}
    \draw (0,0) -- (7,0);
    \begin{scope}[shift={(1.5,0.15)}]
        \draw (0,0) rectangle ++(1.2,0.8) node[pos=0.5] {\SI{1}{kg}};
        \draw[fill=white] (0.3,0) circle (0.15);
        \draw[fill=white] (0.9,0) circle (0.15);
        \draw[thick,->] (0,0.4) -- ++(-0.8,0);
    \end{scope}
    \begin{scope}[shift={(4,0.15)}]
        \draw (0,0) rectangle ++(1.5,1) node[pos=0.5] {\SI{2}{kg}};
        \draw[fill=white] (0.4,0) circle (0.15);
        \draw[fill=white] (1.1,0) circle (0.15);
        \draw[thick,->] (1.5,0.5) -- ++(0.8,0);
    \end{scope}
\end{tikzpicture}
\end{center}

Which cart will have the greatest magnitude of acceleration and why?

\begin{randomizechoices}
    \choice The \SI{1}{kg} cart because the force from the \SI{2}{kg} cart will push a greater amount on it.	
    \correctchoice The \SI{1}{kg} cart because its smaller mass will be affected more by the force from the \SI{2}{kg} cart than the larger \SI{2}{kg} cart will be from the \SI{1}{kg} cart.	
    \choice The \SI{2}{kg} cart because the force from the \SI{1}{kg} cart will push a greater amount on it.
    \choice The \SI{2}{kg} cart because its larger mass will be affected more by the force of the \SI{1}{kg} cart than the smaller \SI{1}{kg} cart will be from the \SI{2}{kg} cart.    
\end{randomizechoices}

\question
A student pushes a 12-kg block on a frictionless, horizontal surface. If the block is initially at rest, what is the speed of the block after the student pushes the block for 5 seconds with an acceleration of \SI{2.0}{m/s^2}?

\begin{randomizechoices}
    \choice \SI{2.0}{m/s}
    \choice \SI{6.0}{m/s}
    \correctchoice \SI{10}{m/s}
    \choice \SI{60}{m/s}
\end{randomizechoices}

\question
A force is applied to a block, causing it to accelerate along a horizontal, frictionless surface. The energy gained by the block is equal to the

\begin{randomizechoices}
    \correctchoice net work done on the block.
    \choice impulse applied to the block.
    \choice momentum given to the block.
    \choice power applied to the block.
\end{randomizechoices}

\question
Two carts collide and bounce off one another. The diagram below shows their motion after the collision.


\begin{center}
\begin{tikzpicture}
    \draw (0,0) -- (7,0);
    \begin{scope}[shift={(1.5,0.15)}]
        \draw (0,0) rectangle ++(1.2,0.8) node[pos=0.5] {\SI{1}{kg}};
        \draw[fill=white] (0.3,0) circle (0.15);
        \draw[fill=white] (0.9,0) circle (0.15);
        \draw[thick,->] (0,0.4) -- ++(-0.8,0);
    \end{scope}
    \begin{scope}[shift={(4,0.15)}]
        \draw (0,0) rectangle ++(1.5,1) node[pos=0.5] {\SI{2}{kg}};
        \draw[fill=white] (0.4,0) circle (0.15);
        \draw[fill=white] (1.1,0) circle (0.15);
        \draw[thick,->] (1.5,0.5) -- ++(0.8,0);
    \end{scope}
\end{tikzpicture}
\end{center}

If the average force during the collision on the \SI{1}{kg} cart from the \SI{2}{kg} cart is \SI{-4}{N}, what is the average force on the \SI{2}{kg} cart from the \SI{1}{kg} cart?

\begin{randomizechoices}
    \correctchoice \SI{4}{N}
    \choice \SI{8}{N}
    \choice \SI{2}{N}
    \choice \SI{1}{N}
\end{randomizechoices}

\clearpage

\question
A car has an oil drip. As the car moves in the positive direction, it drips oil at a regular rate, leaving a trail of spots on the road. The oil drop pattern is shown in the figure below.

\begin{center}
\begin{tikzpicture}[scale=0.9]
    \draw[domain=-10:0,mark=*,only marks, samples=8,mark size=3pt] plot({0.1*\x^2},0);
    % \draw[thick, <-] (-5,0.5) -- ++(-1,0) node[above,pos=0.5] {$\vec{v}$};
    \node[above=3pt] at (-10,0) {Start};
    \node[above=3pt] at (0,0) {Finish};
\end{tikzpicture}
\end{center}

Choose the description that best matches the motion of the car.

\begin{randomizechoices}
    \choice The car is slowing down with a positive acceleration.
    \choice The car is speeding up with a positive acceleration.
    \choice The car is speeding up with a negative acceleration.
    \correctchoice The car is slowing down with a negative acceleration.
\end{randomizechoices}

\question
Four forces act on an object as shown in the free body diagram below. $F_f$ is equal in magnitude to $F$. $F_\mathrm{N}$ is equal in magnitude to $F_g$.

\begin{center}
\begin{tikzpicture}
    \draw (0,0) -- (5,0);
    \draw[thick,fill=black!10] (1.5,0) rectangle ++(1,1);
    \fill (2,0.5) circle (2pt);
    \draw[->] (2,0.5) -- ++(0,+1.3) node[above] {$F_\mathrm{N}$};
    \draw[->] (2,0.5) -- ++(0,-1.3) node[below] {$F_g$};
    \draw[->] (2,0.5) -- ++(+1.5,0) node[right] {$F$};
    \draw[->] (2,0.5) -- ++(-1.5,0) node[left] {$F_f$};
\end{tikzpicture}
\end{center}

Which graph below best represents the velocity of this object?

\begin{center}
\begin{tikzpicture}
    \begin{axis}[height=4cm,
        width=4cm,
        ymin=-3,ymax=3,
        xmin=0,xmax=3,
        ticks=none,
        axis y line=left,
        axis x line=center,
        ylabel={Velocity},
        % y label style={rotate=-90},
        xlabel={$t$},
        clip=false,
        title={Graph A}
    ]
        \addplot[domain=0:2.5,very thick,black] {-x+2.5};
        \draw (0,-3) node[below,white] {$t$}; %...FOR ALIGNMENT
    \end{axis}
\end{tikzpicture}
\hspace{2em}
\begin{tikzpicture}
    \begin{axis}[height=4cm,
        width=4cm,
        ymin=-3,ymax=3,
        xmin=0,xmax=3,
        ticks=none,
        axis y line=left,
        axis x line=center,
        ylabel={Velocity},
        % y label style={rotate=-90},
        xlabel={Time},
        clip=false,
        title={Graph B}
    ]
        \addplot[domain=0:2.5,very thick,black] {1.5};
        \draw (0,-3) node[below,white] {$t$}; %...FOR ALIGNMNET
    \end{axis}
\end{tikzpicture}
\hspace{2em}
\begin{tikzpicture}
    \begin{axis}[height=4cm,
        width=4cm,
        ymin=-3,ymax=3,
        xmin=0,xmax=3,
        ticks=none,
        axis y line=left,
        axis x line=center,
        ylabel={Velocity},
        % y label style={rotate=-90},
        xlabel={Time},
        clip=false,
        title={Graph C}
    ]
        \addplot[domain=0:2.5,very thick,black] {-1.5 + 1.5*x};
        \draw (0,-3) node[below,white] {$t$}; %...FOR ALIGNMNET
    \end{axis}
\end{tikzpicture}
\hspace{2em}
\begin{tikzpicture}
    \begin{axis}[height=4cm,
        width=4cm,
        ymin=-3,ymax=3,
        xmin=0,xmax=3,
        ticks=none,
        axis y line=left,
        axis x line=center,
        ylabel={Velocity},
        % y label style={rotate=-90},
        xlabel={Time},
        clip=false,
        title={Graph D}
    ]
        \addplot[domain=0:2.5,very thick,black] {x};
        \draw (0,-3) node[below,white] {$t$}; %...FOR ALIGNMNET
    \end{axis}
\end{tikzpicture}
\end{center}

{\color{white} \tiny
\begin{randomizeoneparchoices}[norandomize]
    \choice Graph A
    \correctchoice Graph B
    \choice Graph C
    \choice Graph D
\end{randomizeoneparchoices}
}

\question
A constant braking force of \SI{10}{N} is applied for \SI{5}{s} to stop a \SI{2.5}{kg} cart.  Calculate the cart's change in momentum.

\begin{randomizeoneparchoices}
    \choice \SI{12.5}{kg\cdot m/s}
    \choice \SI{25}{kg\cdot m/s}
    \correctchoice \SI{50}{kg\cdot m/s}
    \choice \SI{125}{kg\cdot m/s}
\end{randomizeoneparchoices}

\question
The diagram below shows a \SI{4.0}{kg} object accelerating at \SI{2.0}{m/s^2} on a rough horizontal surface. 

\begin{center}
\begin{tikzpicture}
    \draw (0,0) -- (8,0);
    \begin{scope}[shift={(3,0)}]
        \draw[thick,->] (2,0.5) -- ++(+1.5,0) node[right] {$F_\mathrm{A} = \SI{50}{N}$};
        \draw[thick,->] (0,0.5) -- ++(-1.5,0) node[left] {$F_f$};
        \draw[thick,fill=black!10] (0,0) rectangle ++(2,1.2) node[pos=0.5] {\SI{4.0}{kg}};
    \end{scope}
\end{tikzpicture}
\end{center}

What is the magnitude of the frictional force $F_f$ acting on the object?

\begin{randomizechoices}
    \choice \SI{58}{N}
    \correctchoice \SI{42}{N}
    \choice \SI{2}{N}
    \choice \SI{8}{N}
\end{randomizechoices}

\clearpage
\printkeytable


\end{questions}
\end{document}