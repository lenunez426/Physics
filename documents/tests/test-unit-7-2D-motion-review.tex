\documentclass[answers]{exam}
\usepackage{marvosym}

%...TikZ & PGF
\usepackage{pgfplots}
\pgfplotsset{compat=1.11}
\tikzset{>=latex}
\usetikzlibrary{calc,math}
\usepackage{tikzsymbols}
\usepgfplotslibrary{fillbetween}
\usetikzlibrary{decorations.markings} 
\usetikzlibrary{arrows.meta} %...APP2 for arrows as objects and images
\usetikzlibrary{backgrounds} %...For shading portions of graphs
\usetikzlibrary{patterns} %...Unit 5 Problems
\usetikzlibrary{shapes.geometric} %...For drawing cylinders in Unit 2
\tikzset{
    mark position/.style args={#1(#2)}{
        postaction={
            decorate,
            decoration={
                markings,
                mark=at position #1 with \coordinate (#2);
            }
        }
    }
} %...See https://tex.stackexchange.com/questions/43960/define-node-at-relative-coordinates-of-draw-plot

\tikzset{
    declare function = {trajectoryequation10(\x,\vi,\thetai)= tan(\thetai)*\x - 10*\x^2/(2*(\vi*cos(\thetai))^2);},
    declare function = {trajectoryequation(\x,\vi,\thetai)= tan(\thetai)*\x - 9.8*\x^2/(2*(\vi*cos(\thetai))^2);},
    declare function = {patheq(\x,\yi,\vi,\thetai)= \yi + tan(\thetai)*\x - 9.8*\x^2/(2*(\vi*cos(\thetai))^2);},
    declare function = {patheqten(\x,\yi,\vi,\thetai)= \yi + tan(\thetai)*\x - 10*\x^2/(2*(\vi*cos(\thetai))^2);} %like patheq but with gravity = 10
}

%...siunitx
\usepackage{siunitx}
\DeclareSIUnit{\nothing}{\relax}
\def\mymu{\SI{}{\micro\nothing} }
\DeclareSIUnit\mmHg{mmHg}
\DeclareSIUnit{\mile}{mi}
%...NOTE: "The product symbol between the number and unit is set using the quantity-product option."

%...Other
\usepackage{amsthm}
\usepackage{amsmath}
\usepackage{amssymb}
\usepackage{cancel}
\usepackage{subcaption}
\usepackage{dashrule}
\usepackage{enumitem}
\usepackage{fontawesome}
\usepackage{multicol}
\usepackage{glossaries}
%\numberwithin{equation}{section}
\numberwithin{figure}{section}
\usepackage{float}
\usepackage{twemojis} %...twitter emojis
\usepackage{utfsym}
\newcommand{\R}{\mathbb{R}} %...real number symbol
\usepackage{graphicx}
\graphicspath{ {../Figures/} }
\usepackage{hyperref}
\hypersetup{colorlinks=true,
    linkcolor=blue,
    filecolor=magenta,
    urlcolor=cyan,}
\urlstyle{same}
\newcommand{\hdashline}{{\hdashrule{\textwidth}{0.5pt}{0.8mm}}}
\newcommand{\hgraydashline}{{\color{lightgray} \hdashrule{0.99\textwidth}{1pt}{0.8mm}}}

%...Miscellaneous user-defined symbols
\newcommand{\fnet}{F_{\text{net}}} %...For net force
\newcommand{\bvec}[1]{\vec{\mathbf{#1}}} %...bold vector
\newcommand{\bhat}[1]{\,\hat{\mathbf{#1}}} %...bold hat vector
\newcommand{\que}{\mathord{?}}  %...Question mark symbol in equation env
%...Define thick horizontal rule for examples:
\newcommand{\hhrule}{\hrule\hrule}
\let\oldtexttt\texttt% Store \texttt
\renewcommand{\texttt}[2][black]{\textcolor{#1}{\ttfamily #2}}% 

%...For use in the exam document class
\newif\ifprintmetasolutions


%...Decreases space above and below align and gather enironment
\makeatletter
\g@addto@macro\normalsize{%
  \setlength\abovedisplayskip{-3pt}
  \setlength\belowdisplayskip{6pt} 
}
\makeatother





\usepackage[margin=1in]{geometry}
\usepackage[figurewithin=none]{caption}
\usepackage{exam-randomizechoices}

\CorrectChoiceEmphasis{\color{red}\bfseries}
\renewcommand{\solutiontitle}{\noindent\textbf{\textcolor{red}{Solution:}}\enspace}

\usepackage{OutilsGeomTikz}
\usepackage{utfsym} %...Symbols in Unit 7 Problems
\usepackage{tabu} %...Symbols in Unit 7 Problems

%...For use in Unit 2            %    
\setlength{\columnsep}{2cm}      %
\setlength{\columnseprule}{1pt}  %
\usepackage[none]{hyphenat}      %
%%%%%%%%%%%%%%%%%%%%%%%%%%%%%%%%%

%...For use in Unit 11 on Waves:
\pgfdeclarehorizontalshading{visiblelight}{50bp}{  %
color(0.00000000000000bp)=(red);                   %
color(8.33333333333333bp)=(orange);                %
color(16.66666666666670bp)=(yellow);               %
color(25.00000000000000bp)=(green);                %
color(33.33333333333330bp)=(cyan);                 %
color(41.66666666666670bp)=(blue);                 %
color(50.00000000000000bp)=(violet)                %
}                                                  %

\newcommand{\checkbox}[1]{%
  \ifnum#1=1
    \makebox[0pt][l]{\raisebox{0.15ex}{\hspace{0.1em}\Large$\checkmark$}}%
  \fi
  $\square$%
}
%%%%%%%%%%%%%%%%%%%%%%%%%%%%%%%%%%%%%%%%%%%%%%%%%%%%

%...If using circuitikz package:
% \ctikzset{bipoles/battery1/height=0.5}
% \ctikzset{bipoles/battery1/width=0.25}
% \ctikzset{bipoles/resistor/height=0.15}
% \ctikzset{bipoles/resistor/width=0.4}
\usepackage{mdframed}
\usepackage[none]{hyphenat}

\setrandomizerseed{1}
\bracketedpoints
\addpoints

\newif\ifversionKlevel

\versionKleveltrue

\header{Physics \\Review on Units 6 \& 7: One- \& Two-Dimensional Motion}{}{Name:\enspace\makebox[5cm]{\hrulefill}}

\ifversionKlevel
    \header{Physics \\Review on Units 6 \& 7: One- \& Two-Dimensional Motion}{}{Name:\enspace\makebox[5cm]{\hrulefill}}
\fi

\begin{document}

\clearpage

\subsection*{Review}

\begin{questions}

\question
The JP Morgan Chase Tower is the tallest building in Houston. If you drop a bowling ball from rest from the top of the building, it takes 7.81 seconds for it to strike the ground below. How tall is the building?

\begin{solutionorbox}[4cm]
We know time is $t = \SI{7.81}{s}$ and the magnitude of gravitational acceleration is $g = \SI{10}{m/s^2}$. Since the object is dropped from rest, initial velocity is $v_i = 0$. The ball's vertical displacement is given by the one-dimensional motion equation

\vspace{-1em}
\begin{align*}
    \Delta y &= v_i t - \frac{1}{2}g t^2 \\[1ex]
    &= (\SI{0}{m/s})(\SI{7.81}{s}) - \frac{1}{2} (\SI{10}{m/s^2})(\SI{7.81}{s})^2 \\[1ex]
    &= -\SI{305}{m}
\end{align*}

This vertical displacement must be equal in magnitude to the height of the building. Therefore, the building's height is \SI{305}{m}.    
\end{solutionorbox}

\question
The Burj Kalifa, the tallest building in the world, is 830 meters tall. How long would it take a bowling ball that is horizontally launched at \SI{5}{m/s} from the top of the building to strike the ground?

\medskip

\begin{minipage}{0.25\textwidth}
\begin{tikzpicture}
\begin{axis}[width=6cm,height=6cm,
    axis lines = none,
    clip=false,
    axis equal image,
    ymin=0, ymax=125,
    xmin=0, xmax=90,
    ]
    \addplot[domain=0:90,densely dashed] {trajectoryequation10(x,18,0)+125};    
    \fill[black!10] (0,0) rectangle ++(-20,125);
    \draw (-20,0) -- (0,0) -- (0,125);
    \node at (-4,128) {\Strichmaxerl[1]};
\end{axis}
\end{tikzpicture}
\end{minipage}
\hspace{2mm}%
\fbox{
\begin{minipage}{0.65\textwidth}
\ifprintanswers
\color{red}
\else
\color{white}
\fi
In falling, the ball's displacement is $\Delta y = -\SI{830}{m}$. For a horizontally launched projectile, the vertical component of initial velocity is zero: $v_{iy} = 0$. Gravitational acceleration has a magnitude of $g = \SI{10}{m/s^2}$.

To solve for hang time, we start with the one-dimensional motion equation

\begin{equation*}
    \Delta y = v_{iy} t - \frac{1}{2}g t^2
\end{equation*}

Substitution of known values yields

\begin{equation*}
    -\SI{830}{m} = 0 - \frac{1}{2}\left(\SI{10}{m/s^2}\right) t^2
\end{equation*}

Solving for time gives

\begin{equation*}
    t = \sqrt{\frac{2(\SI{830}{m})}{\SI{10}{m/s^2}}} = \boxed{\SI{12.9}{s}}
\end{equation*}
\end{minipage}
}
\question
A projectile is launched at an angle above the horizon as shown in the figure below. 

\begin{center}
\begin{tikzpicture}
    \begin{axis}[height=6cm,width=9cm,
        axis lines=left,
        ylabel={$y$ (m)},
        y label style={rotate=-90},
        xlabel={$x$ (m)},
        ymin=0,ymax=14,
        xmin=0,xmax=30,
        ytick={0,2,...,14},
        xtick={0,3,...,30},
        grid=both,
        clip=false,
    ]
    \addplot[domain=0:28.06,thick,smooth,densely dashed] {trajectoryequation10(x,18,60)};
    \ifprintanswers
    \draw[red,ultra thick,->] (0,0) -- ({2.5cm*cos(60)},{2.5cm*sin(60)}) node[right=2pt,pos=0.9] {\SI{18}{m/s}};
    \draw[red] (0.7cm,0) arc (0:60:0.7cm) node[red,pos=0.7,right=2pt] {\ang{60}};
    \fill[red] (14.03,12.15) circle (2pt) node[above=2pt] {apex};
    \fi
    \end{axis}
\end{tikzpicture}
\end{center}

Complete the following parts:

{\hfill \textit{\dots continued on next page.}}

\begin{parts}
\part At the origin, draw a velocity vector representing the initial speed. Label it \SI{18}{m/s}.
\part Estimate the launch angle and label it on the graph.
\part Label the apex.
\part The apex height is about \fillin[12][4em] meters.
\part The maximum range is about \fillin[28][4em] meters.
\part Discuss the changes, if any, to the apex height, apex range, and maximum projectile range when the initial speed is increased.

\ifprintanswers
\bgroup
\color{red}
When speed increases, the trajectory grows taller and wider such that the apex height, apex range, and maximum projectile all increase.
\egroup
\else
\fillwithlines{2cm}
\fi
\end{parts}

\question
A pumpkin is launched from a cannon. 

\begin{parts}
\part Draw and label the horizontal component of velocity ($v_x$) and vertical component ($v_y$) at each point in the trajectory below. The directions and relative magnitudes must be accurate.

\begin{center}
\begin{tikzpicture}
    \begin{axis}[height=7cm,width=10cm,
        axis lines=none,
        ylabel={$y$ (m)},
        xlabel={$x$ (m)},
        ymin=0,ymax=12,
        xmin=0,xmax=16,
        ytick={0,2,...,12},
        xtick={0,2,...,16},
        grid=both,
        clip=false,
    ]
    \ifprintanswers
        \draw[->,thick,red] (0,0) -- ++(2,0) node[above] {$v_{ix}$};
        \draw[->,thick,red] (0,0) -- ++(0,5.5) node[above] {$v_{iy}$};
    \fi
    \fill[black] (0,0) circle (2pt);
    \pgfplotsinvokeforeach{0.41,0.81,2.05,2.46,2.87}{
        \ifprintanswers
            \color{red}
        \else
            \color{white}
        \fi
        \coordinate (P) at ({15*cos(70)*#1},{15*sin(70)*#1 - 0.5*9.8*(#1)^2}); 
        \draw[->,thick] (P) -- ++(2,0) node[above] {$v_x$};
        \draw[->,thick] (P) -- ++(0,{5.5 - 3.82*#1}) node[right] {$v_y$};
        \fill[black] (P) circle (2pt);
    }

    %...For apex:
        \coordinate (P) at ({15*cos(70)*1.44},{15*sin(70)*1.44 - 0.5*9.8*1.44^2});
        \draw[->,thick] (P) -- ++(2,0) node[above] {$v_x$};
        \fill[black] (P) circle (2pt);

    %...Trajectory
    \addplot[domain=0:14.74,black,smooth] {trajectoryequation(x,15,70)};
    \end{axis}
\end{tikzpicture}
\end{center}

\part 
Now draw and label $a_x$ and $a_y$, the horizontal and vertical components of acceleration, respectively.

\begin{center}
\begin{tikzpicture}
    \begin{axis}[height=7cm,width=10cm,
        axis lines=none,
        ylabel={$y$ (m)},
        xlabel={$x$ (m)},
        ymin=0,ymax=12,
        xmin=0,xmax=16,
        ytick={0,2,...,12},
        xtick={0,2,...,16},
        grid=both,
        clip=false,
    ]
    \pgfplotsinvokeforeach{0,0.41,0.81,1.44,2.05,2.46,2.87}{
        \ifprintanswers
            \color{red}
        \else
            \color{white}
        \fi
        \coordinate (P) at ({15*cos(70)*#1},{15*sin(70)*#1 - 0.5*9.8*(#1)^2}); 

        \draw[->,thick] (P) -- ++(0,-3) node[below] {$a_y$};
        \fill[black] (P) circle (2pt);
    }

    %...Trajectory
    \addplot[domain=0:14.74,black,smooth] {trajectoryequation(x,15,70)};
    \end{axis}
\end{tikzpicture}
\end{center}
\end{parts}

\clearpage
\question
The figure below shows a fly undergoing circular motion. Draw and label the linear speed vector ($v$) and the centripetal acceleration vector ($a_c$). 

\begin{center}
\begin{tikzpicture}[x=2cm,y=2cm]
    \coordinate (P) at ({cos(135)},{sin(135)});
    \draw[densely dashed] (0,0) circle (1);
    \ifprintanswers
    \draw[thick,->,red] (P) -- ++({0.7*cos(45)},{-0.7*sin(45)}) node[above right=-1pt] {$a_c$};
    \draw[thick,red,->] (P) -- ++({-cos(45)},{-sin(45)}) node[above left=-1pt] {$v$};
    \fi
    \draw (P) node[rotate=135] {\twemoji[width=8mm]{fly}};
\end{tikzpicture}
\end{center}

\question 
Calculate the circumference of the circle below.

\bigskip

\begin{minipage}{0.25\textwidth}
\begin{tikzpicture}[x=2cm,y=2cm]
    \draw (0,0) circle (1);
    \draw (0,0) -- (1,0) node[above,pos=0.5] {\SI{6.0}{cm}};
\end{tikzpicture}
\end{minipage}
\hspace{1cm}%
\fbox{
\begin{minipage}{0.6\textwidth}
\ifprintanswers
\color{red}
\else
\color{white}
\fi
\begin{align*}
    C &= 2\pi r \\[1ex]
    &= 2\pi (\SI{6.0}{cm}) \\[1ex]
    &= \boxed{\SI{37.7}{cm}}
\end{align*}

\vspace{1cm}
\end{minipage}
}

\question
A remote control toy airplane is undergoing uniform circular motion, and its path is shown below. It takes 45 seconds complete one circular trip. Calculate the plane's linear speed.

\bigskip

\begin{minipage}{0.25\textwidth}
\begin{tikzpicture}[x=2cm,y=2cm]
    \draw[densely dashed] (0,0) circle (1);
    \draw (0,0) -- (1,0) node[above,pos=0.5] {\SI{76}{m}};
\end{tikzpicture}
\end{minipage}
\hspace{1cm}%
\fbox{
\begin{minipage}{0.6\textwidth}
\ifprintanswers
\color{red}
\else
\color{white}
\fi
Distance traveled by the plane is

\begin{equation*}
    s = 2\pi r = 2\pi(\SI{76}{m}) = \SI{477.5}{m}
\end{equation*}

Linear speed is distance over time:

\begin{equation*}
    v = \frac{2\pi r}{t} = \frac{2\pi(\SI{76}{m})}{\SI{45}{s}} = \boxed{\SI{10.6}{m/s}}
\end{equation*}

\vspace{5mm}
\end{minipage}
}

\question \label{OXyBFH}
A car turns at \SI{20}{m/s} on a curved road and undergoes circular motion, as shown below. Calculate its centripetal acceleration.

\begin{minipage}{0.3\textwidth}
\centering
\begin{tikzpicture}[x=2cm,y=2cm]
    \draw[densely dashed] (1,0) arc (0:180:1);
    \draw[->,thick] ({cos(45)},{sin(45)}) -- ++({-0.8*cos(45)},{0.8*sin(45)}) node[left,black] {\SI{20}{m/s}};
    \ifprintanswers
    \draw[->,very thick] ({cos(45)},{sin(45)}) -- ++({-0.6*cos(45)},{-0.6*sin(45)}) node[above left=-2pt,black,pos=0.8] {$a_c$};
    \fi
    \fill (0,0) circle (1pt);
    \draw[<->] (0,0) -- (-1,0) node[above,pos=0.5] {\SI{50}{m}};
    \fill ({cos(45)},{sin(45)}) circle (3pt) node[right=3pt] {car};
\end{tikzpicture}
\end{minipage}
\hspace{1em}%
\fbox{
\begin{minipage}{0.6\textwidth}
\ifprintanswers
\color{red}
\else
\color{white}
\fi

\begin{align*}
    a_c &= \frac{v^2}{r} \\[1ex]\
        &= \frac{(\SI{20}{m/s})^2}{\SI{50}{m}} \\[1ex]
        &= \boxed{\SI{8.0}{m/s^2}}
\end{align*}

\vspace{5mm}
\end{minipage}
}

\clearpage

\begin{EnvUplevel}
\textbf{Questions \ref{2nF3qF}--\ref{oSRba9}: Refer to the prompt and figure below.}

A volleyball is tethered to a pole by a 2.0-meter rope and is moving with circular motion. The figure below shows the bird's eye view of the system.

\begin{center}
\begin{tikzpicture}[x=2cm,y=2cm]
    \coordinate (P) at ({cos(45)},{-sin(45)});
    \draw[densely dashed] (0,0) circle (1);
    \draw[thick] (0,0) -- (P);
    \ifprintanswers
    \draw[very thick,->,red] (P) -- ++({-0.7*cos(45)},{0.7*sin(45)}) node[right=2mm] {$F_c$};
    \draw[thick,red,->] (P) -- ++({cos(45)},{sin(45)}) node[above left=-1pt] {$v$};
    \fi
    \node at (P) {\twemoji[width=8mm]{volleyball}};
\end{tikzpicture}
\end{center}    
\end{EnvUplevel}

\question \label{2nF3qF}
In the figure above,

\begin{parts}
    \part draw and label the centripetal force vector on the ball.
    \part draw the path the volleyball would take if the rope suddenly snapped off at the instant shown. 
\end{parts}


\question
The volleyball stays in circular motion due to the centripetal force $F_c$. What type of force is $F_c$?

\begin{randomizechoices}
    \correctchoice tension force
    \choice gravitational force
    \choice friction force
    \choice magnetic force
\end{randomizechoices}

\question \label{oSRba9}
The volleyball has a mass of 260 grams. If it takes 3.0 seconds to complete one trip around the circle, what is the centripetal force on the ball?

\begin{solutionorbox}[4cm]
The linear speed is

\begin{equation*}
    v = \frac{2\pi r}{t} = \frac{2\pi(\SI{2.0}{m})}{\SI{3.0}{s}} = \SI{4.19}{m/s}
\end{equation*}

The centripetal force is

\begin{equation*}
    F_c = \frac{mv^2}{r} = \frac{(\SI{0.26}{kg})\left(\SI{4.19}{m/s}\right)^2}{\SI{2.0}{m}} = \boxed{\SI{2.28}{N}}
\end{equation*}
\end{solutionorbox}

\bigskip

\hrule

\question \label{}
A ball of mass $m$ tethered to a pole by a 6.0-meter rope is moving with circular motion. The centripetal force on the ball is $F$, and the ball moves with linear speed $v$. If the centripetal force increases to $\frac{3}{2}F$, what new length of rope must be used to keep the ball's speed at $v$?

\bigskip

\begin{minipage}{0.3\textwidth}
\centering
\begin{tikzpicture}[x=2cm,y=2cm]
    \draw[densely dashed] (1,0) arc (0:180:1);
    \fill (0,0) circle (1pt);
    \draw[thick] (0,0) -- ({cos(45)},{sin(45)}) node {\twemoji[width=8mm]{basketball}} node[pos=0.5,below right] {$r$};
\end{tikzpicture}
\end{minipage}
\hspace{1em}%
\fbox{
\begin{minipage}{0.6\textwidth}
\ifprintanswers
\color{red}
\else
\color{white}
\fi

\textbf{Answer:} 4.0 meters

\bigskip

See solution below.
\vspace{2cm}
\end{minipage}
}

\begin{solution}
Let the original and new centripetal forces be

\begin{equation*}
    F_1 = \frac{mv^2}{r} \quad \text{and} \quad F_2 = \frac{mv^2}{r_2} = \frac{3}{2} F_1
\end{equation*}

The ratio of forces is

\begin{equation*}
    \frac{F_2}{F_1} = \frac{\displaystyle \left(\frac{mv^2}{r_2}\right)}{\left(\displaystyle \frac{mv^2}{r}\right)} = \frac{r}{r_2}
\end{equation*}

We're also given

\begin{equation*}
    \frac{F_2}{F_1} = \frac{\frac{3}{2}F_1}{F_1} = \frac{3}{2}
\end{equation*}

So,

\begin{equation*}
    \frac{r}{r_2} = \frac{3}{2}
\end{equation*}

Solving for the new length $r_2$, we get

\begin{equation*}
    r_2 = \frac{2}{3}r = \frac{2}{3} (\SI{6.0}{m}) = \boxed{\SI{4.0}{m}}
\end{equation*}
\end{solution}

\end{questions}


\end{document}



% \clearpage

% \question
% The combined mass of Cody and his car, from Question \ref{OXyBFH}, is \SI{740}{kg}. Calculate the centripetal force on the car.

% \begin{randomizechoices}
%     \correctchoice \SI{2770}{N}
%     \choice \SI{178}{N}
%     \choice \SI{3083}{N}
%     \choice \SI{200400}{N}
% \end{randomizechoices}