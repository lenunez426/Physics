\documentclass[answers]{exam}
\usepackage{marvosym}

%...TikZ & PGF
\usepackage{pgfplots}
\pgfplotsset{compat=1.11}
\tikzset{>=latex}
\usetikzlibrary{calc,math}
\usepackage{tikzsymbols}
\usepgfplotslibrary{fillbetween}
\usetikzlibrary{decorations.markings} 
\usetikzlibrary{arrows.meta} %...APP2 for arrows as objects and images
\usetikzlibrary{backgrounds} %...For shading portions of graphs
\usetikzlibrary{patterns} %...Unit 5 Problems
\usetikzlibrary{shapes.geometric} %...For drawing cylinders in Unit 2
\tikzset{
    mark position/.style args={#1(#2)}{
        postaction={
            decorate,
            decoration={
                markings,
                mark=at position #1 with \coordinate (#2);
            }
        }
    }
} %...See https://tex.stackexchange.com/questions/43960/define-node-at-relative-coordinates-of-draw-plot

\tikzset{
    declare function = {trajectoryequation10(\x,\vi,\thetai)= tan(\thetai)*\x - 10*\x^2/(2*(\vi*cos(\thetai))^2);},
    declare function = {trajectoryequation(\x,\vi,\thetai)= tan(\thetai)*\x - 9.8*\x^2/(2*(\vi*cos(\thetai))^2);},
    declare function = {patheq(\x,\yi,\vi,\thetai)= \yi + tan(\thetai)*\x - 9.8*\x^2/(2*(\vi*cos(\thetai))^2);},
    declare function = {patheqten(\x,\yi,\vi,\thetai)= \yi + tan(\thetai)*\x - 10*\x^2/(2*(\vi*cos(\thetai))^2);} %like patheq but with gravity = 10
}

%...siunitx
\usepackage{siunitx}
\DeclareSIUnit{\nothing}{\relax}
\def\mymu{\SI{}{\micro\nothing} }
\DeclareSIUnit\mmHg{mmHg}
\DeclareSIUnit{\mile}{mi}
%...NOTE: "The product symbol between the number and unit is set using the quantity-product option."

%...Other
\usepackage{amsthm}
\usepackage{amsmath}
\usepackage{amssymb}
\usepackage{cancel}
\usepackage{subcaption}
\usepackage{dashrule}
\usepackage{enumitem}
\usepackage{fontawesome}
\usepackage{multicol}
\usepackage{glossaries}
%\numberwithin{equation}{section}
\numberwithin{figure}{section}
\usepackage{float}
\usepackage{twemojis} %...twitter emojis
\usepackage{utfsym}
\newcommand{\R}{\mathbb{R}} %...real number symbol
\usepackage{graphicx}
\graphicspath{ {../Figures/} }
\usepackage{hyperref}
\hypersetup{colorlinks=true,
    linkcolor=blue,
    filecolor=magenta,
    urlcolor=cyan,}
\urlstyle{same}
\newcommand{\hdashline}{{\hdashrule{\textwidth}{0.5pt}{0.8mm}}}
\newcommand{\hgraydashline}{{\color{lightgray} \hdashrule{0.99\textwidth}{1pt}{0.8mm}}}

%...Miscellaneous user-defined symbols
\newcommand{\fnet}{F_{\text{net}}} %...For net force
\newcommand{\bvec}[1]{\vec{\mathbf{#1}}} %...bold vector
\newcommand{\bhat}[1]{\,\hat{\mathbf{#1}}} %...bold hat vector
\newcommand{\que}{\mathord{?}}  %...Question mark symbol in equation env
%...Define thick horizontal rule for examples:
\newcommand{\hhrule}{\hrule\hrule}
\let\oldtexttt\texttt% Store \texttt
\renewcommand{\texttt}[2][black]{\textcolor{#1}{\ttfamily #2}}% 

%...For use in the exam document class
\newif\ifprintmetasolutions


%...Decreases space above and below align and gather enironment
\makeatletter
\g@addto@macro\normalsize{%
  \setlength\abovedisplayskip{-3pt}
  \setlength\belowdisplayskip{6pt} 
}
\makeatother





\usepackage[margin=1in]{geometry}
\usepackage[figurewithin=none]{caption}
\usepackage{exam-randomizechoices}

\CorrectChoiceEmphasis{\color{red}\bfseries}
\renewcommand{\solutiontitle}{\noindent\textbf{\textcolor{red}{Solution:}}\enspace}

\usepackage{OutilsGeomTikz}
\usepackage{utfsym} %...Symbols in Unit 7 Problems
\usepackage{tabu} %...Symbols in Unit 7 Problems

%...For use in Unit 2            %    
\setlength{\columnsep}{2cm}      %
\setlength{\columnseprule}{1pt}  %
\usepackage[none]{hyphenat}      %
%%%%%%%%%%%%%%%%%%%%%%%%%%%%%%%%%

%...For use in Unit 11 on Waves:
\pgfdeclarehorizontalshading{visiblelight}{50bp}{  %
color(0.00000000000000bp)=(red);                   %
color(8.33333333333333bp)=(orange);                %
color(16.66666666666670bp)=(yellow);               %
color(25.00000000000000bp)=(green);                %
color(33.33333333333330bp)=(cyan);                 %
color(41.66666666666670bp)=(blue);                 %
color(50.00000000000000bp)=(violet)                %
}                                                  %

\newcommand{\checkbox}[1]{%
  \ifnum#1=1
    \makebox[0pt][l]{\raisebox{0.15ex}{\hspace{0.1em}\Large$\checkmark$}}%
  \fi
  $\square$%
}
%%%%%%%%%%%%%%%%%%%%%%%%%%%%%%%%%%%%%%%%%%%%%%%%%%%%

%...If using circuitikz package:
% \ctikzset{bipoles/battery1/height=0.5}
% \ctikzset{bipoles/battery1/width=0.25}
% \ctikzset{bipoles/resistor/height=0.15}
% \ctikzset{bipoles/resistor/width=0.4}
\usepackage{mdframed}

\setrandomizerseed{1}


\header{Physics\\Review for TEKS Test}{}{Name:\enspace\makebox[5cm]{\hrulefill}}


\begin{document}
\begin{questions}
\question 
A satellite is launched into frictionless space. True or false? Once in space, the amount of force needed to keep the satellite moving at a constant speed is at least 1 newton. Explain

\ifprintanswers
\else
\fillwithlines{2cm}
\fi

\begin{solution}
    According to the law of inertia, once in motion the satellite will remain in motion at a constant velocity. Therefore, no force is required to keep it in motion. The statement is false. 
\end{solution}

\question
For each diagram below, the box is the identical and the force $F$ is applied by various angles above the horizontal. In Diagrams A and C, force $F$ is applied at \ang{45} from the horizontal.

\begin{center}
\begin{tikzpicture}
    \draw (0,0) -- (3,0) node[below,pos=0.5] {Diagram A};
    \begin{scope}[shift={(1,0)}]
        \draw[fill=black!10] (0,0) rectangle (1,0.7);
        \draw[thick,->] (1,0.7) -- ++(1,0.6) node[above] {$F$};        
    \end{scope}
\end{tikzpicture}
\hspace{1cm}
\begin{tikzpicture}
    \draw (0,0) -- (3,0) node[below,pos=0.5] {Diagram B};
    \begin{scope}[shift={(1,0)}]
        \draw[fill=black!10] (0,0) rectangle (1,0.7);
        \draw[thick,->] (1,0.35) -- ++(1.17,0) node[above] {$F$};        
    \end{scope}
\end{tikzpicture}
\hspace{1cm}
\begin{tikzpicture}
    \draw (0,0) -- (3,0) node[below,pos=0.5] {Diagram C};
    \begin{scope}[shift={(1,0)}]
        \draw[fill=black!10] (0,0) rectangle (1,0.7);
        \draw[thick,<-] (1,0.7)node[above=2pt] {$F$} -- ++(1,0.6);        
    \end{scope}
\end{tikzpicture}
\end{center}

Rank order the magnitude of acceleration of the box from least to greatest. Assume all surfaces are frictionless.

\begin{solution}
    The magnitudes of acceleration are equal for Diagrams A and C. The greatest acceleration occurs for Diagram B.
\end{solution}

\question
% The bar chart below shows the kinetic energy of a 3-kg object at two different points in time.

% \begin{center}
% \begin{tikzpicture}
% \begin{axis}[width=7cm,height=6cm,
%     % axis lines=left,
%     ybar,
%     symbolic x coords={initial, final},
%     xtick=data,
%     ylabel={Kinetic Energy (J)},
%     ymin=0, ymax=10,
%     ytick={0,1,...,10},
%     bar width=35pt,
%     enlarge x limits=0.5,
%     ymajorgrids=true,
%     tick pos=left,
%     title={Change in Kinetic Energy}
% ]
% \addplot[black,fill=black!20] coordinates {(initial,8) (final,2)};
% \end{axis}
% \end{tikzpicture}
% \end{center}

The work-energy theorem states that the net work done on an object is equal to 

\begin{randomizechoices}
    \correctchoice change in kinetic energy
    \choice net force multiplied by time
    \choice mass multiplied by acceleration
    \choice change in momentum 
\end{randomizechoices}

\question
(a) Draw a velocity vs. time graph for a car traveling at a constant velocity of \SI{2}{m/s} for 4 seconds.\\ (b) Discuss how you can use the graph to calculate the car's displacement during that time interval.

\begin{center}
\begin{tikzpicture}
    \begin{axis}[width=8cm,height=5cm,
        axis lines=left,
        ymin=0, ymax=5,
        xmin=0, xmax=6,
        ylabel = {Velocity (m/s)},
        xlabel = {Time (s)},
        grid = both,
        xtick={0,1,...,10},
        ytick={0,1,...,6},
        minor x tick num=1,
    ]
    \ifprintanswers
    \addplot[very thick,domain=0:4,red] plot {2};
    \fi
\end{axis}
\end{tikzpicture}
\end{center}



\question
A storm chaser flying a drone at \SI{21}{m/s} accelerates it at \SI{2.0}{m/s^2} for \SI{5.0}{s}. What is the drone's speed at the end of this time interval?


\begin{solution}
\begin{equation*}
    a = \frac{\Delta v}{\Delta t} = \frac{v_f - v_0}{\Delta t}
\end{equation*}

\vspace{-1ex}
\begin{align*}
    v_f &= v_0 + a \Delta t \\[1ex]
    &= \SI{21}{m/s} + \left(\SI{2.0}{m/s^2}\right)(\SI{5.0}{s}) \\[1ex]
    &= \boxed{\SI{31}{m/s}}
\end{align*}
\end{solution}

\question
The velocity vs. time data table below represents the motion of an object.

\begin{center}
    \begin{tabular}{|c|c|}
        \hline
         \textbf{Time} (s) & \textbf{Velocity} (m/s) \\ \hline
         0 & 0 \\ \hline
         1 & 1 \\ \hline
         2 & 2 \\ \hline
         3 & 3 \\ \hline
         4 & 4 \\ \hline
         5 & 5 \\ \hline
         6 & 6 \\ \hline
    \end{tabular}
\end{center}

What additional information would you need to find the force acting on the object during this time interval?

\ifprintanswers
\else
\fillwithlines{2cm}
\fi

\begin{solution}
    You can use the table to find the acceleration. By Newton's 2nd law, force is mass multiplied by acceleration. Therefore, we additionally need to know the mass of the object. 
\end{solution}


% \question
% The chart below is a record of the instantaneous velocities at the given times for a toy car used in a laboratory investigation.

% \begin{center}
%     \begin{tabular}{|c|c|}
%         \hline
%          \textbf{Time} (s) & \textbf{Velocity} (m/s) \\ \hline
%          0 & $+3$ \\ \hline
%          1 & $+1$ \\ \hline
%          2 & $-1$ \\ \hline
%          3 & $-3$ \\ \hline
%     \end{tabular}
% \end{center}

% Which of the following statements is supported by the data?

% \begin{randomizechoices}
%     \choice From 0 seconds to 1 seconds, the car is speeding up.
%     \choice From 1 second to 2 seconds, the car has a constant velocity.
%     \choice From 2 seconds to 3 seconds, the car is slowing down.
%     \correctchoice From 0 seconds to 3 seconds, the velocity is changing at a constant rate.
% \end{randomizechoices}

\question
The data represented in the position vs. time graph below was collected from a cart moving along a horizontal track.

\begin{center}
\begin{tikzpicture}
    \begin{axis}[height=5cm,width=7cm,
        axis lines=left,
        ymin=0, ymax=80,
        xmin=0, xmax=6,
        ylabel = {Position (m)},
        xlabel = {Time (s)},
        grid = both,
        ytick={0,20,...,80},
        xtick={0,1,...,6},
        clip=false,
    ]
    \addplot[very thick] coordinates {(0,40)(1,40)(2,60)(4,60)(5,0)}; 
\end{axis}
\end{tikzpicture}
\end{center}

During which time interval(s) is the cart not moving?

\ifprintanswers
\else
\fillwithlines{1cm}
\fi

\begin{solution}
    Between 0 and 1 second, and between 2 and 4 seconds, the car does not move.
\end{solution}

% \begin{randomizechoices}
%     \choice Between 0 to 1 seconds	
%     \choice Between 1 to 2 seconds
%     \choice Between 2 to 4 seconds
%     \correctchoice Between 4 to 5 seconds
% \end{randomizechoices}

\question
An air bag is used to safely decrease the momentum of a driver in a car accident. Use the impulse-momentum theorem to explain how the air bag reduces the magnitude of the force acting on the driver.

\ifprintanswers
\else
\fillwithlines{2cm}
\fi

\begin{solution}
    According to the impulse-momentum theorem, $F_\mathrm{net} \Delta t = \Delta p$, an airbag reduces the force on the driver by increasing the amount of time the force acts on the driver to reduce his momentum to zero.
\end{solution}

% \begin{randomizechoices}	
%     \choice decreasing the distance over which the force acts on the driver.	
%     \choice increasing the rate of acceleration of the driver.	
%     \choice decreasing the mass of the driver.	
%     \correctchoice increasing the amount of time the force acts on the driver.
% \end{randomizechoices}

\question
The object illustrated below is moving horizontally. 

\begin{center}
\begin{tikzpicture}
    \draw (0,0) -- (5,0);
    \draw[thick,fill=black!10] (1.5,0) rectangle ++(1,1);
    \fill (2,0.5) circle (2pt);
    \draw[->] (2,0.5) -- ++(0,+1.3) node[above] {$F_\mathrm{N}$};
    \draw[->] (2,0.5) -- ++(0,-1.3) node[below] {$F_g$};
    \draw[->] (2,0.5) -- ++(+2.0,0) node[above,pos=0.9] {$+$} node[right] {$F_\mathrm{A}$};
    \draw[->] (2,0.5) -- ++(-1.5,0) node[left] {$F_f$};
\end{tikzpicture}
\end{center}

The magnitude of the applied force $F_\mathrm{A}$ is greater the magnitude of the frictional force $F_f$. The object must be moving with 

\begin{randomizechoices}
    \correctchoice constant, positive acceleration.
    \choice constant, positive velocity.
    \choice constant, negative velocity.
    \choice constant, negative acceleration.
\end{randomizechoices}

\question
The velocity vs. time graph below represents the motion of a 4 kg cart along a straight line.

\begin{center}
\begin{tikzpicture}
    \begin{axis}[width=7cm,height=5cm,
        axis lines=left,
        ymin=0, ymax=7,
        xmin=0, xmax=4,
        ylabel = {Velocity (m/s)},
        xlabel = {Time (s)},
        grid = both,
        ytick={0,1,...,7},
        xtick={0,1,...,4},
        minor x tick num=1,
    ]
    \addplot[ultra thick] coordinates {(0,6)(1,6)(1.25,3)(3.5,3)};
\end{axis}
\end{tikzpicture}
\end{center}

Explain why the change in momentum of the cart between $t = 2$ and $t = 3$ seconds is \SI{0}{\kg\cdot m/s}.

\ifprintanswers
\else
\fillwithlines{2cm}
\fi

\begin{solution}
Change in momentum ($\Delta p$) is mass ($m$) multiplied by change in velocity ($\Delta v$). During this time interval, the cart undergoes no change in velocity and therefore does not change momentum.
\end{solution}



% \question
% The graph below shows the force exerted on a box sliding across the floor with a positive velocity.

% \begin{center}
% \begin{tikzpicture}
%     \begin{axis}[width=7cm,height=5cm,
%         axis lines=left,
%         ymin=0, ymax=60,
%         xmin=0, xmax=30,
%         ylabel = {Force (N)},
%         xlabel = {Displacement (m)},
%         grid = both,
%         ytick={0,10,...,60},
%         xtick={0,5,...,30},
%         grid=both
%     ]
%     \addplot[ultra thick] coordinates {(0,40)(20,40)(30,0)};
% \end{axis}
% \end{tikzpicture}
% \end{center}

% What is the change in kinetic energy of the box as it slides through a displacement of \SI{15}{m} with a \SI{40}{N} net force acting on it?



\question
Two carts collide and bounce off one another. The diagram below shows their motion after the collision.


\begin{center}
\begin{tikzpicture}
    \draw (0,0) -- (7,0);
    \begin{scope}[shift={(1.5,0.15)}]
        \draw (0,0) rectangle ++(1.2,0.8) node[pos=0.5] {\SI{1}{kg}};
        \draw[fill=white] (0.3,0) circle (0.15);
        \draw[fill=white] (0.9,0) circle (0.15);
        \draw[thick,->] (0,0.4) -- ++(-0.8,0);
    \end{scope}
    \begin{scope}[shift={(4,0.15)}]
        \draw (0,0) rectangle ++(1.5,1) node[pos=0.5] {\SI{2}{kg}};
        \draw[fill=white] (0.4,0) circle (0.15);
        \draw[fill=white] (1.1,0) circle (0.15);
        \draw[thick,->] (1.5,0.5) -- ++(0.8,0);
    \end{scope}
\end{tikzpicture}
\end{center}

During the collision, which cart exerted a greater force on the other? Explain.

\ifprintanswers
\else
\fillwithlines{2cm}
\fi

\begin{solution}
    Because the collision is a force-pair interaction, the magnitudes of forces exerted by each cart on the other during the collision is equal in magnitude, even if the carts have different masses.
\end{solution}


\question
A car has an oil drip. As the car moves in the positive direction, it drips oil at a regular rate, leaving a trail of spots on the road. Draw the oil drop pattern for a car speeding up with a a positive acceleration. (\textit{Hint:} Draw the motion map.)

\begin{center}
\begin{tikzpicture}[scale=0.9]
    \ifprintanswers
    \draw[domain=-10:0,mark=*,only marks, samples=8,mark size=3pt,red] plot({-0.1*\x^2 -10},0);
    \else
    \draw[domain=-10:0,mark=*,only marks, samples=8,mark size=3pt,white] plot({-0.1*\x^2 -10},0);
    \fi
    \node[above=3pt] at (-10,0) {Start};
    \node[above=3pt] at (0,0) {Finish};
\end{tikzpicture}
\end{center}

\question
Four forces act on an object as shown. $F_f$ is equal in magnitude to $F$. $F_\mathrm{N}$ is equal in magnitude to $F_g$. 
Assuming the block is in motion, make a graph of its position vs. time.


\begin{center}
\begin{tikzpicture}
    \draw (0,0) -- (5,0);
    \draw[thick,fill=black!10] (1.5,0) rectangle ++(1,1);
    \fill (2,0.5) circle (2pt);
    \draw[->] (2,0.5) -- ++(0,+1.3) node[above] {$F_\mathrm{N}$};
    \draw[->] (2,0.5) -- ++(0,-1.3) node[below] {$F_g$};
    \draw[->] (2,0.5) -- ++(+1.5,0) node[right] {$F$};
    \draw[->] (2,0.5) -- ++(-1.5,0) node[left] {$F_f$};
\end{tikzpicture}
\hspace{1cm}
\begin{tikzpicture}
    \begin{axis}[height=5cm,
        width=5cm,
        ymin=-3,ymax=3,
        xmin=0,xmax=3,
        ticks=none,
        axis y line=left,
        axis x line=center,
        ylabel={Position (m)},
        % y label style={rotate=-90},
        xlabel={Time},
        clip=false,
    ]
        \ifprintanswers
        \addplot[domain=0:2.5,very thick,red] {x};
        \fi
    \end{axis}
\end{tikzpicture}
\end{center}







\end{questions}
\end{document}