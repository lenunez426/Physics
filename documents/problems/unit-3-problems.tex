% \documentclass[answers]{exam}
% \usepackage{marvosym}

%...TikZ & PGF
\usepackage{pgfplots}
\pgfplotsset{compat=1.11}
\tikzset{>=latex}
\usetikzlibrary{calc,math}
\usepackage{tikzsymbols}
\usepgfplotslibrary{fillbetween}
\usetikzlibrary{decorations.markings} 
\usetikzlibrary{arrows.meta} %...APP2 for arrows as objects and images
\usetikzlibrary{backgrounds} %...For shading portions of graphs
\usetikzlibrary{patterns} %...Unit 5 Problems
\usetikzlibrary{shapes.geometric} %...For drawing cylinders in Unit 2
\tikzset{
    mark position/.style args={#1(#2)}{
        postaction={
            decorate,
            decoration={
                markings,
                mark=at position #1 with \coordinate (#2);
            }
        }
    }
} %...See https://tex.stackexchange.com/questions/43960/define-node-at-relative-coordinates-of-draw-plot

\tikzset{
    declare function = {trajectoryequation10(\x,\vi,\thetai)= tan(\thetai)*\x - 10*\x^2/(2*(\vi*cos(\thetai))^2);},
    declare function = {trajectoryequation(\x,\vi,\thetai)= tan(\thetai)*\x - 9.8*\x^2/(2*(\vi*cos(\thetai))^2);},
    declare function = {patheq(\x,\yi,\vi,\thetai)= \yi + tan(\thetai)*\x - 9.8*\x^2/(2*(\vi*cos(\thetai))^2);},
    declare function = {patheqten(\x,\yi,\vi,\thetai)= \yi + tan(\thetai)*\x - 10*\x^2/(2*(\vi*cos(\thetai))^2);} %like patheq but with gravity = 10
}

%...siunitx
\usepackage{siunitx}
\DeclareSIUnit{\nothing}{\relax}
\def\mymu{\SI{}{\micro\nothing} }
\DeclareSIUnit\mmHg{mmHg}
\DeclareSIUnit{\mile}{mi}
%...NOTE: "The product symbol between the number and unit is set using the quantity-product option."

%...Other
\usepackage{amsthm}
\usepackage{amsmath}
\usepackage{amssymb}
\usepackage{cancel}
\usepackage{subcaption}
\usepackage{dashrule}
\usepackage{enumitem}
\usepackage{fontawesome}
\usepackage{multicol}
\usepackage{glossaries}
%\numberwithin{equation}{section}
\numberwithin{figure}{section}
\usepackage{float}
\usepackage{twemojis} %...twitter emojis
\usepackage{utfsym}
\newcommand{\R}{\mathbb{R}} %...real number symbol
\usepackage{graphicx}
\graphicspath{ {../Figures/} }
\usepackage{hyperref}
\hypersetup{colorlinks=true,
    linkcolor=blue,
    filecolor=magenta,
    urlcolor=cyan,}
\urlstyle{same}
\newcommand{\hdashline}{{\hdashrule{\textwidth}{0.5pt}{0.8mm}}}
\newcommand{\hgraydashline}{{\color{lightgray} \hdashrule{0.99\textwidth}{1pt}{0.8mm}}}

%...Miscellaneous user-defined symbols
\newcommand{\fnet}{F_{\text{net}}} %...For net force
\newcommand{\bvec}[1]{\vec{\mathbf{#1}}} %...bold vector
\newcommand{\bhat}[1]{\,\hat{\mathbf{#1}}} %...bold hat vector
\newcommand{\que}{\mathord{?}}  %...Question mark symbol in equation env
%...Define thick horizontal rule for examples:
\newcommand{\hhrule}{\hrule\hrule}
\let\oldtexttt\texttt% Store \texttt
\renewcommand{\texttt}[2][black]{\textcolor{#1}{\ttfamily #2}}% 

%...For use in the exam document class
\newif\ifprintmetasolutions


%...Decreases space above and below align and gather enironment
\makeatletter
\g@addto@macro\normalsize{%
  \setlength\abovedisplayskip{-3pt}
  \setlength\belowdisplayskip{6pt} 
}
\makeatother





% \usepackage[margin=1in]{geometry}
\usepackage[figurewithin=none]{caption}
\usepackage{exam-randomizechoices}

\CorrectChoiceEmphasis{\color{red}\bfseries}
\renewcommand{\solutiontitle}{\noindent\textbf{\textcolor{red}{Solution:}}\enspace}

\usepackage{OutilsGeomTikz}
\usepackage{utfsym} %...Symbols in Unit 7 Problems
\usepackage{tabu} %...Symbols in Unit 7 Problems

%...For use in Unit 2            %    
\setlength{\columnsep}{2cm}      %
\setlength{\columnseprule}{1pt}  %
\usepackage[none]{hyphenat}      %
%%%%%%%%%%%%%%%%%%%%%%%%%%%%%%%%%

%...For use in Unit 11 on Waves:
\pgfdeclarehorizontalshading{visiblelight}{50bp}{  %
color(0.00000000000000bp)=(red);                   %
color(8.33333333333333bp)=(orange);                %
color(16.66666666666670bp)=(yellow);               %
color(25.00000000000000bp)=(green);                %
color(33.33333333333330bp)=(cyan);                 %
color(41.66666666666670bp)=(blue);                 %
color(50.00000000000000bp)=(violet)                %
}                                                  %

\newcommand{\checkbox}[1]{%
  \ifnum#1=1
    \makebox[0pt][l]{\raisebox{0.15ex}{\hspace{0.1em}\Large$\checkmark$}}%
  \fi
  $\square$%
}
%%%%%%%%%%%%%%%%%%%%%%%%%%%%%%%%%%%%%%%%%%%%%%%%%%%%

%...If using circuitikz package:
% \ctikzset{bipoles/battery1/height=0.5}
% \ctikzset{bipoles/battery1/width=0.25}
% \ctikzset{bipoles/resistor/height=0.15}
% \ctikzset{bipoles/resistor/width=0.4}
% %\usetikzlibrary{backgrounds} %...For shading portions of graphs

% \setrandomizerseed{1}

% \firstpageheader{Physics}{Unit 3: Changing Motion}{Problems}
% \runningheader{Physics}{}{Unit 3 Problems}

% \setlength{\columnsep}{2cm}
% \setlength{\columnseprule}{1pt}
% \usepackage[none]{hyphenat}

\documentclass[../main-physics-problems.tex]{subfiles}

\begin{document}
\subsection{Acceleration}
\subsubsection{The Concept of Acceleration}
\subsubsection{Zero Acceleration, Positive Acceleration, and Negative Acceleration}
\subsubsection{Comparing Accelerations of Objects}
\subsubsection{Determining Change in Velocity and Acceleration from Multiple Representations}
\subsubsection{Representations of Changing Motion}
\subsubsection{Predicting Change in Velocity and Acceleration using Number Sense}
\subsection{Newton’s Law of Acceleration}
\subsubsection{The Relationship between Net Force and Acceleration}
\subsubsection{The Relationship between Mass and Acceleration}
\subsubsection{The Effects of Changing Net Force on Mass or Acceleration}
\subsubsection{Solving Problems using Newton’s Second Law}

\subsection*{Representing Acceleration Ramp Lab}

\begin{questions}
\question
Place the motion sensor at the top of the ramp, and place the cart in front of the sensor.

\begin{center}
    \begin{tikzpicture}
        \begin{scope}[rotate=-10,transform shape]
            \draw (0,0) rectangle ++(10,-0.5) ;%node[pos=0.12,below=-2pt] {0 position} node[pos=0.98,above] {$+$};
            \draw[above=9mm] (3,0) rectangle ++(1.5,-0.7) node[pos=0.5,below=-3mm] {cart};
            \draw[fill=white,above=2mm] (3.4,0) circle (2mm);
            \draw[fill=white,above=2mm] (4.1,0) circle (2mm);
            \draw[fill=black!15,above=1pt] (0,0) rectangle (1,1.5) node[pos=0.5,above=7.5mm,align=left] {motion\\detector};
            % \draw (9,0) node[above=2mm,align=left] {Highest point};
            % \draw[thick,->] (3,1.2) -- ++(2,0) node[above,pos=0.5] {Initial push};
        \end{scope}
    \end{tikzpicture}
\end{center}
\end{questions}

\begin{questions}
\begin{EnvUplevel}
    \subsection*{Interpreting Accelerated Motion on Graphs}
\end{EnvUplevel}

\question
Choose the position time graph that shows an object slowing down towards the origin.


\begin{center}
    \begin{tikzpicture}
        \begin{axis}[height=4cm,
            width=4cm,
            ymin=0,ymax=5,
            xmin=0,xmax=3,
            ticks=none,
            axis lines=left,
            ylabel={$x$},
            y label style={rotate=-90},
            xlabel={$t$},
            title=Graph A,
        ]
            \addplot[domain=0:2,very thick,black] {4*x - x^2};
        \end{axis}
    \end{tikzpicture}
    \hspace{2em}
    \begin{tikzpicture}
        \begin{axis}[height=4cm,
            width=4cm,
            ymin=0,ymax=5,
            xmin=0,xmax=3,
            ticks=none,
            axis lines=left,
            ylabel={$x$},
            y label style={rotate=-90},
            xlabel={$t$},
            title={Graph B},
        ]
            \addplot[domain=0:2,very thick,black] {x^2};
        \end{axis}
    \end{tikzpicture}
    \hspace{2em}
    \begin{tikzpicture}
        \begin{axis}[height=4cm,
            width=4cm,
            ymin=0,ymax=5,
            xmin=0,xmax=3,
            ticks=none,
            axis lines=left,
            ylabel={$x$},
            y label style={rotate=-90},
            xlabel={$t$},
            title={Graph C}
        ]
            \addplot[domain=0:2,very thick,black] {4*(x+2) - (x+2)^2};
        \end{axis}
    \end{tikzpicture}
    \hspace{2em}
    \begin{tikzpicture}
        \begin{axis}[height=4cm,
            width=4cm,
            ymin=0,ymax=5,
            xmin=0,xmax=3,
            ticks=none,
            axis lines=left,
            ylabel={$x$},
            y label style={rotate=-90},
            xlabel={$t$},
            title={Graph D}
        ]
            \addplot[domain=0:2,very thick,black] {(x-2)^2};
        \end{axis}
    \end{tikzpicture}
\end{center}

{\color{white}
\begin{randomizeoneparchoices}[norandomize]
    \choice Graph A
    \choice Graph B
    \choice Graph C
    \correctchoice Graph D
\end{randomizeoneparchoices}
}


\question
Choose the velocity vs time graph that matches the position graph below.

\begin{center}
\begin{tikzpicture}
    \begin{axis}[height=4cm,
        width=6cm,
        ymin=0,ymax=5,
        xmin=0,xmax=2.5,
        ticks=none,
        axis lines=left,
        ylabel={$x$},
        y label style={rotate=-90},
        xlabel={$t$},
    ]
        \addplot[domain=0:2,very thick,black] {4*x - x^2};
    \end{axis}
\end{tikzpicture}
\end{center}


\begin{center}
\begin{tikzpicture}
    \begin{axis}[height=4cm,
        width=4cm,
        ymin=-3,ymax=3,
        xmin=0,xmax=3,
        ticks=none,
        axis y line=left,
        axis x line=center,
        ylabel={$v$},
        y label style={rotate=-90},
        xlabel={$t$},
        clip=false,
        title={Graph A}
    ]
        \addplot[domain=0:2.5,very thick,black] {-x};
        \draw (0,-3) node[below,white] {$t$}; %...FOR ALIGNMNET
    \end{axis}
\end{tikzpicture}
\hspace{2em}
\begin{tikzpicture}
    \begin{axis}[height=4cm,
        width=4cm,
        ymin=-3,ymax=3,
        xmin=0,xmax=3,
        ticks=none,
        axis y line=left,
        axis x line=center,
        ylabel={$v$},
        y label style={rotate=-90},
        xlabel={$t$},
        clip=false,
        title={Graph B}
    ]
        \addplot[domain=0:2.5,very thick,black] {x-2.5};
        \draw (0,-3) node[below,white] {$t$}; %...FOR ALIGNMNET
    \end{axis}
\end{tikzpicture}
\hspace{2em}
\begin{tikzpicture}
    \begin{axis}[height=4cm,
        width=4cm,
        ymin=-3,ymax=3,
        xmin=0,xmax=3,
        ticks=none,
        axis y line=left,
        axis x line=center,
        ylabel={$v$},
        y label style={rotate=-90},
        xlabel={$t$},
        clip=false,
        title={Graph C}
    ]
        \addplot[domain=0:2.5,very thick,black] {x};
        \draw (0,-3) node[below,white] {$t$}; %...FOR ALIGNMNET
    \end{axis}
\end{tikzpicture}
\hspace{2em}
\begin{tikzpicture}
    \begin{axis}[height=4cm,
        width=4cm,
        ymin=-3,ymax=3,
        xmin=0,xmax=3,
        ticks=none,
        axis y line=left,
        axis x line=center,
        ylabel={$v$},
        y label style={rotate=-90},
        xlabel={$t$},
        clip=false,
        title={Graph D}
    ]
        \addplot[domain=0:2.5,very thick,black] {-x+2.5};
        \draw (0,-3) node[below,white] {$t$}; %...FOR ALIGNMNET
    \end{axis}
\end{tikzpicture}
\end{center}

{\color{white}
\begin{randomizeoneparchoices}[norandomize]
    \choice Graph A
    \choice Graph B
    \choice Graph C
    \correctchoice Graph D
\end{randomizeoneparchoices}
}

\question
Choose the position vs time graph that matches the velocity graph below.

\begin{center}
\begin{tikzpicture}
    \begin{axis}[height=4cm,
        width=6cm,
        ymin=-3,ymax=3,
        xmin=0,xmax=3,
        ticks=none,
        axis y line=left,
        axis x line=center,
        ylabel={$v$},
        y label style={rotate=-90},
        xlabel={$t$},
        clip=false,
    ]
        \addplot[domain=0:2.5,very thick,black] {-x};
        \draw (0,-3) node[below,white] {$t$}; %...FOR ALIGNMNET
    \end{axis}
\end{tikzpicture}
\end{center}

\begin{center}
    \begin{tikzpicture}
        \begin{axis}[height=4cm,
            width=4cm,
            ymin=0,ymax=5,
            xmin=0,xmax=3,
            ticks=none,
            axis lines=left,
            ylabel={$x$},
            y label style={rotate=-90},
            xlabel={$t$},
            title=Graph A,
        ]
            \addplot[domain=0:2,very thick,black] {4*x - x^2};
        \end{axis}
    \end{tikzpicture}
    \hspace{2em}
    \begin{tikzpicture}
        \begin{axis}[height=4cm,
            width=4cm,
            ymin=0,ymax=5,
            xmin=0,xmax=3,
            ticks=none,
            axis lines=left,
            ylabel={$x$},
            y label style={rotate=-90},
            xlabel={$t$},
            title={Graph B},
        ]
            \addplot[domain=0:2,very thick,black] {x^2};
        \end{axis}
    \end{tikzpicture}
    \hspace{2em}
    \begin{tikzpicture}
        \begin{axis}[height=4cm,
            width=4cm,
            ymin=0,ymax=5,
            xmin=0,xmax=3,
            ticks=none,
            axis lines=left,
            ylabel={$x$},
            y label style={rotate=-90},
            xlabel={$t$},
            title={Graph C}
        ]
            \addplot[domain=0:2,very thick,black] {4*(x+2) - (x+2)^2};
        \end{axis}
    \end{tikzpicture}
    \hspace{2em}
    \begin{tikzpicture}
        \begin{axis}[height=4cm,
            width=4cm,
            ymin=0,ymax=5,
            xmin=0,xmax=3,
            ticks=none,
            axis lines=left,
            ylabel={$x$},
            y label style={rotate=-90},
            xlabel={$t$},
            title={Graph D}
        ]
            \addplot[domain=0:2,very thick,black] {(x-2)^2};
        \end{axis}
    \end{tikzpicture}
\end{center}

{\color{white}
\begin{randomizeoneparchoices}[norandomize]
    \choice Graph A
    \choice Graph B
    \correctchoice Graph C
    \choice Graph D
\end{randomizeoneparchoices}
}

\question
Draw the position vs time graph and the velocity vs time graph for the situation shown below:

\begin{center}
    \begin{tikzpicture}
        \begin{scope}[rotate=-10,transform shape]
            \draw (0,0) rectangle ++(10,-0.5) node[pos=0.1,below=-2pt] {0 position} node[pos=0.98,above] {$+$};
            \draw[above=9mm] (8,0) rectangle ++(1.5,-0.7) node[pos=0.5,below=-3mm] {cart};
            \draw[fill=white,above=2mm] (8.4,0) circle (2mm);
            \draw[fill=white,above=2mm] (9.1,0) circle (2mm);
            \draw[ultra thick,->,above=6mm] (8,0) -- ++(-2,0);
        \end{scope}
    \end{tikzpicture}
\end{center}

\begin{center}
\begin{tikzpicture}
    \begin{axis}[height=5cm,
        width=7cm,
        ymin=0,ymax=5,
        xmin=0,xmax=2.5,
        ticks=none,
        axis lines=left,
        ylabel={$x$},
        y label style={rotate=-90},
        xlabel={$t$},
    ]
        % \addplot[domain=0:2,very thick,black] {4*x - x^2};
    \end{axis}
\end{tikzpicture}
\hspace{1cm}
\begin{tikzpicture}
    \begin{axis}[height=5cm,
        width=7cm,
        ymin=-3,ymax=3,
        xmin=0,xmax=3,
        ticks=none,
        axis y line=left,
        axis x line=center,
        ylabel={$v$},
        y label style={rotate=-90},
        xlabel={$t$},
        clip=false,
    ]
        % \addplot[domain=0:2.5,very thick,black] {-x};
        \draw (0,-3) node[below,white] {$t$}; %...FOR ALIGNMNET
    \end{axis}
\end{tikzpicture}
\end{center}

\question \label{riRu3}
The graph below represents the motion of an object. Describe the motion of the object during each segment---from 0 to 1 second, 1 to 3 seconds, and 3 to 5 seconds---using one of the items in the list. 

\begin{minipage}{0.45\textwidth}
\centering
\begin{tikzpicture}
    \begin{axis}[height=5cm,
        width=7cm,
        ymin=0,ymax=4.0,
        xmin=0,xmax=5.0,
        ytick={0,1,...,4},
        minor y tick num=1,
        xtick={0,1,...,5},
        axis lines=left,
        ylabel={$x$},
        y label style={rotate=-90},
        xlabel={$t$},
        grid=both,
    ]
        \addplot[very thick,black] coordinates{(0,0) (1,3) (3,3)};
        \addplot[very thick,black,domain=3:5] {3 - 0.5*(x-3)^2};
    \end{axis}
\end{tikzpicture}
\end{minipage}%
\hspace{2em}
\begin{minipage}{0.45\textwidth}
    \centering
\begin{itemize}[itemsep=0pt]
    \item Constant velocity, away from origin
    \item Constant velocity, towards origin
    \item At rest/stopped
    \item Speeding up, away from origin
    \item Speeding up, towards origin
    \item Slowing down, away from origin
    \item Slowing down, towards origin
\end{itemize}
\end{minipage}


\question \label{Xszkg}
Where is the object from Question \ref{riRu3} speeding up?

\begin{randomizechoices}[norandomize]
    \choice From 0 to 1\,s
    \choice From 1 to 3\,s
    \correctchoice From 3 to 5\,s
\end{randomizechoices}

\clearpage
\question 
Which of the following pair of graphs demonstrates an object moving away from the reference point with a decreasing velocity?

\bigskip

\begin{minipage}{0.45\textwidth}
\centering
\textbf{Pair A}

\vspace{1ex}

\begin{tikzpicture}
    \begin{axis}[height=4cm,
        width=4cm,
        ymin=0,ymax=5,
        xmin=0,xmax=3,
        ticks=none,
        axis lines=left,
        ylabel={$x$},
        y label style={rotate=-90},
        xlabel={$t$},
    ]
        \addplot[domain=0:2,very thick,black] {(x-2)^2};
    \end{axis}
\end{tikzpicture}
\begin{tikzpicture}
    \begin{axis}[height=4cm,
        width=4cm,
        ymin=-3,ymax=3,
        xmin=0,xmax=3,
        ticks=none,
        axis y line=left,
        axis x line=center,
        ylabel={$v$},
        y label style={rotate=-90},
        xlabel={$t$},
        clip=false
    ]
        \addplot[domain=0:2.5,very thick,black] {-x};
        \draw (0,-3) node[below,white] {$t$}; %...FOR ALIGNMNET
    \end{axis}
\end{tikzpicture}
\end{minipage}%
\begin{minipage}{0.45\textwidth}
\centering
\textbf{Pair B}

\vspace{1ex}

\begin{tikzpicture}
    \begin{axis}[height=4cm,
        width=4cm,
        ymin=0,ymax=5,
        xmin=0,xmax=3,
        ticks=none,
        axis lines=left,
        ylabel={$x$},
        y label style={rotate=-90},
        xlabel={$t$},
    ]
        \addplot[domain=0:2,very thick,black] {4*x - x^2};
    \end{axis}
\end{tikzpicture}
\begin{tikzpicture}
    \begin{axis}[height=4cm,
        width=4cm,
        ymin=-3,ymax=3,
        xmin=0,xmax=3,
        ticks=none,
        axis y line=left,
        axis x line=center,
        ylabel={$v$},
        y label style={rotate=-90},
        xlabel={$t$},
        clip=false
    ]
        \addplot[domain=0:2.5,very thick,black] {-x+2.5};
        \draw (0,-3) node[below,white] {$t$}; %...FOR ALIGNMNET
    \end{axis}
\end{tikzpicture}
\end{minipage}

\vspace{1em}

\begin{minipage}{0.45\textwidth}
\centering
\textbf{Pair C}

\vspace{1ex}

\begin{tikzpicture}
    \begin{axis}[height=4cm,
        width=4cm,
        ymin=0,ymax=5,
        xmin=0,xmax=3,
        ticks=none,
        axis lines=left,
        ylabel={$x$},
        y label style={rotate=-90},
        xlabel={$t$},
    ]
        \addplot[domain=0:2,very thick,black] {4*x-x^2};
    \end{axis}
\end{tikzpicture}
\begin{tikzpicture}
    \begin{axis}[height=4cm,
        width=4cm,
        ymin=-3,ymax=3,
        xmin=0,xmax=3,
        ticks=none,
        axis y line=left,
        axis x line=center,
        ylabel={$v$},
        y label style={rotate=-90},
        xlabel={$t$},
        clip=false
    ]
        \addplot[domain=0:2.5,very thick,black] {-x};
        \draw (0,-3) node[below,white] {$t$}; %...FOR ALIGNMNET
    \end{axis}
\end{tikzpicture}
\end{minipage}%
\begin{minipage}{0.45\textwidth}
\centering
\textbf{Pair D}

\vspace{1ex}

\begin{tikzpicture}
    \begin{axis}[height=4cm,
        width=4cm,
        ymin=0,ymax=5,
        xmin=0,xmax=3,
        ticks=none,
        axis lines=left,
        ylabel={$x$},
        y label style={rotate=-90},
        xlabel={$t$},
    ]
        \addplot[domain=0:2,very thick,black] {(x-2)^2};
    \end{axis}
\end{tikzpicture}
\begin{tikzpicture}
    \begin{axis}[height=4cm,
        width=4cm,
        ymin=-3,ymax=3,
        xmin=0,xmax=3,
        ticks=none,
        axis y line=left,
        axis x line=center,
        ylabel={$v$},
        y label style={rotate=-90},
        xlabel={$t$},
        clip=false
    ]
        \addplot[domain=0:2.5,very thick,black] {-x+2.5};
        \draw (0,-3) node[below,white] {$t$}; %...FOR ALIGNMNET
    \end{axis}
\end{tikzpicture}
\end{minipage}


{\color{white}
\begin{randomizeoneparchoices}[norandomize,]
    \choice Pair A
    \correctchoice Pair B
    \choice Pair C
    \choice Pair D
\end{randomizeoneparchoices}
}

\question 
Which of the following velocity vs time graphs demonstrates an object speeding up toward the reference point?

\begin{center}
    \begin{tikzpicture}
        \begin{axis}[height=5cm,
            width=5cm,
            ymin=0,ymax=5,
            xmin=0,xmax=3,
            ticks=none,
            axis lines=left,
            ylabel={position},
            %y label style={rotate=-90},
            xlabel={time},
        ]
            \addplot[domain=0:2,very thick,black] {4*(x+2) - (x+2)^2};
        \end{axis}
    \end{tikzpicture}
\end{center}

\vspace{1ex}

\begin{center}
\begin{tikzpicture}
    \begin{axis}[height=4cm,
        width=4cm,
        ymin=-3,ymax=3,
        xmin=0,xmax=3,
        ticks=none,
        axis y line=left,
        axis x line=center,
        ylabel={$v$},
        y label style={rotate=-90},
        xlabel={$t$},
        clip=false,
        title={Graph A}
    ]
        \addplot[domain=0:2.5,very thick,black] {x-2.5};
        \draw (0,-3) node[below,white] {$t$}; %...FOR ALIGNMNET
    \end{axis}
\end{tikzpicture}
\hspace{2em}
\begin{tikzpicture}
    \begin{axis}[height=4cm,
        width=4cm,
        ymin=-3,ymax=3,
        xmin=0,xmax=3,
        ticks=none,
        axis y line=left,
        axis x line=center,
        ylabel={$v$},
        y label style={rotate=-90},
        xlabel={$t$},
        clip=false,
        title={Graph B}
    ]
        \addplot[domain=0:2.5,very thick,black] {x};
        \draw (0,-3) node[below,white] {$t$}; %...FOR ALIGNMNET
    \end{axis}
\end{tikzpicture}
\hspace{2em}
\begin{tikzpicture}
    \begin{axis}[height=4cm,
        width=4cm,
        ymin=-3,ymax=3,
        xmin=0,xmax=3,
        ticks=none,
        axis y line=left,
        axis x line=center,
        ylabel={$v$},
        y label style={rotate=-90},
        xlabel={$t$},
        clip=false,
        title={Graph C}
    ]
        \addplot[domain=0:2.5,very thick,black] {-x};
        \draw (0,-3) node[below,white] {$t$}; %...FOR ALIGNMNET
    \end{axis}
\end{tikzpicture}
\hspace{2em}
\begin{tikzpicture}
    \begin{axis}[height=4cm,
        width=4cm,
        ymin=-3,ymax=3,
        xmin=0,xmax=3,
        ticks=none,
        axis y line=left,
        axis x line=center,
        ylabel={$v$},
        y label style={rotate=-90},
        xlabel={$t$},
        clip=false,
        title={Graph D}
    ]
        \addplot[domain=0:2.5,very thick,black] {-x+2.5};
        \draw (0,-3) node[below,white] {$t$}; %...FOR ALIGNMNET
    \end{axis}
\end{tikzpicture}
\end{center}

{\color{white}
\begin{randomizeoneparchoices}[norandomize,]
    \choice Graph A
    \choice Graph B
    \correctchoice Graph C
    \choice Graph D
\end{randomizeoneparchoices}
}

\clearpage

\begin{EnvUplevel}
    \subsection*{Calculating Acceleration from a v-t Graph}
\end{EnvUplevel}

\question
What is the acceleration of the motion shown in the graph below?

\begin{center}
\begin{tikzpicture}
    \begin{axis}[height=6cm,width=6cm,
        axis y line=left,
        axis x line=center,
        ymin=-15,ymax=15,
        xmin=0,xmax=25,
        ylabel={$v$ (m/s)},
        xlabel={$t$ (s)},
        grid=both,
        xtick={0,5,...,25},
        ytick={-15,-10,...,15},
        x label style={at={(axis description cs: 1,0.5)},anchor=west}
    ]
        \addplot[very thick,black,domain=0:20] {10-0.5*x};
    \end{axis}
\end{tikzpicture}
\end{center}

\begin{randomizechoices}
    \choice \SI{-2}{m/s^2}
    \correctchoice \SI{-0.5}{m/s^2} 
    \choice \SI{2}{m/s^2}
    \choice \SI{0.5}{m/s^2}
\end{randomizechoices}

\question
What is the acceleration of the motion shown in the graph below?

\begin{center}
\begin{tikzpicture}
    \begin{axis}[height=5cm,width=6cm,
        axis lines=left,
        ymin=0,ymax=5,
        xmin=0,xmax=6,
        ylabel={Velocity (m/s)},
        xlabel={Time (s)},
        grid=both,
        xtick={0,1,...,6},
        ytick={0,1,...,5},
    ]
        \addplot[very thick,black,domain=0:5] {x};
    \end{axis}
\end{tikzpicture}
\end{center}

\begin{randomizechoices}
    \choice \SI{-5}{m/s^2}
    \choice \SI{-1}{m/s^2} 
    \choice \SI{5}{m/s^2}
    \correctchoice \SI{1}{m/s^2}
\end{randomizechoices}

\question
Calculate the acceleration of the data graphed below.

\begin{center}
\begin{tikzpicture}
    \begin{axis}[height=5cm,width=6cm,
        axis lines=left,
        ymin=0,ymax=50,
        xmin=0,xmax=5,
        ylabel={Velocity (m/s)},
        xlabel={Time (s)},
        grid=both,
        ytick={0,10,...,50},
        xtick={0,1,...,5},
    ]
        \addplot[very thick,black,domain=0:5] {10*x};
    \end{axis}
\end{tikzpicture}
\end{center}

\begin{randomizechoices}
    \choice \SI{0.1}{m/s^2}
    \choice \SI{-10}{m/s^2} 
    \choice \SI{45}{m/s^2}
    \correctchoice \SI{10}{m/s^2}
\end{randomizechoices}

\begin{EnvUplevel}
    \textbf{Questions \ref{c1CMp}--\ref{uwahd}. Consider the motions of two objects shown below.}
\end{EnvUplevel}

\begin{center}
\begin{tikzpicture}
    \begin{axis}[height=5cm,width=6cm,
        axis lines=left,
        ymin=0,ymax=30,
        xmin=0,xmax=10,
        ylabel={Velocity (m/s)},
        xlabel={Time (s)},
        grid=both,
        ytick={0,5,...,30},
        xtick={0,2,...,10},
        title={Object A},
    ]
        \addplot[very thick,black,domain=0:10] {2*x+4};
    \end{axis}
\end{tikzpicture}
\hspace{2em}
\begin{tikzpicture}
    \begin{axis}[height=5cm,width=6cm,
        axis y line=left,
        axis x line=center,
        ymin=-10,ymax=0,
        xmin=0,xmax=2,
        ylabel={Velocity (m/s)},
        xlabel={Time (s)},
        grid=both,
        ytick={-10,-8,...,0},
        xtick={0,0.4,...,2},
        x label style={anchor=west},
        % x tick label style={yshift=1mm,anchor=south},
        title={Object B},
        clip=false,
    ]
        \addplot[very thick,black,domain=0:2] {-5*x};
        \node[below=2mm,align=center] at (0,-10) {M\\M};
    \end{axis}
\end{tikzpicture}
\end{center}

\question \label{c1CMp}
What is the acceleration of Object A?

\begin{randomizechoices}
    \choice \SI{-2}{m/s^2}
    \correctchoice \SI{2}{m/s^2}
    \choice \SI{0.4}{m/s^2}
    \choice \SI{-0.4}{m/s^2}
\end{randomizechoices}

\question 
What is the acceleration of the Object B?

\begin{randomizechoices}
    \choice \SI{5}{m/s^2}
    \correctchoice \SI{-5}{m/s^2}
    \choice \SI{0.2}{m/s^2}
    \choice \SI{-0.2}{m/s^2}
\end{randomizechoices}

\question \label{uwahd}
Which object is changing speed at a greater rate?

\begin{randomizechoices}
    \choice Object A, since the object is moving away from the reference point.
    \correctchoice Object B, since it has a steeper slope.
    \choice Object A, since the acceleration is positive.
    \choice Object B, since the acceleration is negative.
\end{randomizechoices}

\bigskip
\hrule

\question
Which object has the greatest acceleration?

\begin{center}
\begin{tikzpicture}
    \begin{axis}[height=6cm,width=7cm,
        axis y line=left,
        axis x line=center,
        ymin=-10,ymax=10,
        xmin=0,xmax=25,
        ylabel={Velocity (m/s)},
        xlabel={Time (s)},
        grid=both,
        ytick={-10,-5,...,10},
        xtick={0,5,...,25},
        x label style={at={(axis description cs: 1,0.5)},anchor=west}
    ]
        \addplot[very thick,black,domain=0:23] {0.44*x} node[below,pos=0.9] {A};
        \addplot[very thick,black,domain=0:23] {0.22*x} node[below,pos=0.9] {B};
    \end{axis}
\end{tikzpicture}
\end{center}

\begin{randomizechoices}
    \choice Object B because it is closer to the time axis.
    \correctchoice Object A because the velocity changes more for any given change in time.
    \choice Object B because its velocity changes more for a any given change in time.
    \choice Object A because its slope is zero.
\end{randomizechoices}

\question
Consider the motion of an electric car depicted below.

\begin{center}
\begin{tikzpicture}
    \begin{axis}[height=5cm,width=7cm,
        axis lines=left,
        ymin=0,ymax=10,
        xmin=0,xmax=18,
        ylabel={Velocity (m/s)},
        xlabel={Time (s)},
        grid=both,
        ytick={0,2,...,10},
        xtick={0,2,...,18},
    ]
        \addplot[very thick,black] coordinates{(0,0)(4,8)(10,8)(18.0,0)};
    \end{axis}
\end{tikzpicture}
\end{center}

Draw the acceleration vs time graph that corresponds to the car's motion.

\begin{center}
\begin{tikzpicture}
    \begin{axis}[height=5cm,width=8cm,
        axis y line=left,
        axis x line=center,
        ymin=-3,ymax=3,
        xmin=0,xmax=18,
        ylabel={Acceleration (\SI{}{m/s^2})},
        xlabel={Time (s)},
        grid=both,
        ytick={-3,-2,...,3},
        xtick={0,2,...,18},
        x label style={at={(axis description cs: 1,0.5)},anchor=west}
    ]
    \end{axis}
\end{tikzpicture}
\end{center}


\clearpage
\begin{EnvUplevel}
    \subsection*{Acceleration Calculations}
\end{EnvUplevel}

\question
An airplane went from 120\,m/s to 180\,m/s in 4.0 seconds. What was its acceleration?

\begin{randomizechoices}
    \choice \SI{30}{m/s^2}
    \correctchoice \SI{15}{m/s^2}
    \choice \SI{60}{m/s^2}
    \choice \SI{45}{m/s^2}
\end{randomizechoices}

\question
A car slows down from 14\,m/s to 6\,m/s in 3 seconds. What is the car's acceleration?

\begin{randomizechoices}
    \choice \SI{6.00}{m/s^2}
    \correctchoice \SI{-2.67}{m/s^2}
    \choice \SI{8.00}{m/s^2}
    \choice \SI{2.67}{m/s^2}
\end{randomizechoices}

\question
A car traveling at a speed of 30\,m/s comes to a stoplight.  How much time will it take for the car to stop if its
acceleration is \SI{-4.0}{m/s^2}?

\begin{randomizechoices}
    \choice \SI{-7.5}{s}
    \choice \SI{1.3}{s}
    \choice \SI{0.75}{s}
    \correctchoice \SI{7.5}{s}
\end{randomizechoices}

\begin{EnvUplevel}
    \textbf{Questions \ref{rruYR}--\ref{ynGeg}.} Consider the graph below.
\end{EnvUplevel}

\begin{center}
\begin{tikzpicture}
    \begin{axis}[width=8cm,height=6cm,
        axis y line=left,
        axis x line=center,
        ylabel={Velocity (m/s)},
        xlabel={Time (s)},
        x label style={at={(axis description cs: 1,0.5)},anchor=west},
        ymin=-3,ymax=3,
        xmin=0,xmax=8,
        ytick={-3,-2,...,3},
        xtick={0,1,...,8},
        grid=both,
    ]
        \addplot[thick,mark=*] coordinates{(0,1)(0.8,1)(0.8)(2,-2)(3,2)(6,2)(7,0.6)(7,0)};
        \begin{pgfonlayer}{background}
            \fill[black!12] (3,0) rectangle (6,2);
        \end{pgfonlayer}
    \end{axis}
\end{tikzpicture}
\end{center}

\question \label{rruYR}
What is the velocity of the object at $t = \SI{4}{s}$?

\begin{randomizechoices}
    \correctchoice \SI{2}{m/s}
    \choice \SI{1}{m/s}
    \choice \SI{-2}{m/s}
    \choice \SI{0}{m/s}
\end{randomizechoices}

\question
What does the area of the shaded box between 3 and 6 seconds represent?

\begin{randomizechoices}
    \choice velocity
    \choice acceleration
    \choice position
    \correctchoice displacement
\end{randomizechoices}

\question \label{ynGeg}
What is the acceleration of the object from 2 to 3 seconds?

\begin{randomizechoices}
    \correctchoice \SI{4}{m/s^2}
    \choice \SI{2}{m/s^2}
    \choice \SI{-2}{m/s^2}
    \choice \SI{0}{m/s^2}
\end{randomizechoices}

\bigskip
\hrule

\question
The graph shows the acceleration of a car over time. 

\begin{center}
\begin{tikzpicture}
    \begin{axis}[height=4cm,width=6cm,
        axis lines=left,
        ylabel={Acceleration (\SI{}{m/s^2})},
        xlabel={Time (s)},
        ymin=0,ymax=5,
        xmin=0,xmax=10,
        ytick={0,1,...,5},
        xtick={0,1,...,10},
        grid=both,
    ]
        \addplot[very thick,black,domain=0:10] {4};
    \end{axis}
\end{tikzpicture}
\end{center}

If the car starts from rest, what is the velocity of the car after 5
seconds?

\begin{randomizechoices}
    \choice \SI{0.80}{m/s}
    \correctchoice \SI{20}{m/s}
    \choice \SI{4.0}{m/s}
    \choice \SI{0}{m/s}
\end{randomizechoices}

\question
Study the data table below.

\begin{center}
    \begin{tabular}{c|c}
        \textbf{Time} (s) & \textbf{Position} (m) \\ \hline
         0 & 0 \\
         1 & 8 \\
         2 & 16 \\
         3 & 24 \\
    \end{tabular}
\end{center}

What is the acceleration of the object?

\begin{randomizechoices}
 \choice \SI{32}{m/s^2}
 \choice \SI{16}{m/s^2}
 \correctchoice \SI{0}{m/s^2}
 \choice \SI{8}{m/s^2}
\end{randomizechoices}

\question 
A child riding a bicycle at 15\,m/s accelerates at \SI{3.0}{m/s^2} for 4.0\,s. What is the child's speed at the end of this 4.0\,s
interval?

\begin{randomizechoices}
    \choice \SI{12}{m/s}
    \correctchoice \SI{27}{m/s}
    \choice \SI{3}{m/s}
    \choice \SI{7}{m/s}
\end{randomizechoices}

\question
The data represented in the velocity vs. time graph below is from a car traveling along a flat straight road.

\begin{center}
\begin{tikzpicture}
    \begin{axis}[height=5cm,width=7cm,
        axis lines=left,
        ylabel={Velocity (m/s)},
        xlabel={Time (s)},
        ymin=0,ymax=30,
        xmin=0,xmax=5,
        ytick={0,10,...,30},
        xtick={0,1,...,5},
        grid=both,
    ]
        \addplot[very thick,black] coordinates{(0,0)(1,10)(5,30)};
    \end{axis}
\end{tikzpicture}
\end{center}

What is the total displacement of the car between 1 second and 5 seconds?

\begin{randomizechoices}
    \choice \SI{100}{m}
    \correctchoice \SI{80}{m}
    \choice \SI{70}{m}
    \choice \SI{8}{m}
\end{randomizechoices}

\question
A student pushes a 12-kg block on a frictionless, horizontal surface. If the block is initially at rest, what is the speed of
the block after the student pushes the block for 5 seconds with an acceleration of \SI{2.0}{m/s^2}?

\begin{randomizechoices}
    \choice \SI{2}{m/s}
    \choice \SI{6}{m/s}
    \correctchoice \SI{10}{m/s}
    \choice \SI{60}{m/s}
\end{randomizechoices}

\question
The car is moving to the right and slowing down. 

\begin{center}
    \begin{tikzpicture}
        \foreach \i in {0,5,7.5,8.75}{
            \node at (\i,0) {\reflectbox{\twemoji[width=8mm]{automobile}}};
        }
    \end{tikzpicture}    
\end{center}

The velocity is \fillin[positive][4cm] and the acceleration is \fillin[negative][4cm].

\clearpage
\begin{EnvUplevel}
    \subsection*{Net Force = Mass times Acceleration}
\end{EnvUplevel}

\question
When the net force applied on a object increases, the acceleration will

\begin{randomizechoices}[keeplast]
    \choice decrease
    \correctchoice increase
    \choice remain the same
\end{randomizechoices}

\question
What is the net force exerted on a 90.0\,kg race-car driver while the race car is accelerating from rest to 44.7\,m/s in 4.50\,s?

\begin{randomizechoices}
    \choice 9.8\,N
    \choice 20\,N
    \choice 201\,N
    \correctchoice 894\,N
\end{randomizechoices}

\question
A heavy football player collides with a light football player. How do the forces they experience compare?

\begin{randomizechoices}
    \choice The light football player feels the greatest force.
    \choice The heavy football player feels the greatest force.
    \correctchoice Each player feels the same amount of force.
\end{randomizechoices}

\question
A spring scale is used to exert a net force of 2.7\,N on a cart. If the cart's mass is 0.64\,kg, what is the cart's acceleration?

\begin{solution}
By Newton's second law,

\begin{equation*}
    a = \frac{F_\mathrm{net}}{m} = \boxed{\SI{4.22}{m/s^2}}
\end{equation*}
\end{solution}

\question
A tow rope is pulling a 1450\,kg truck at \SI{2.2}{m/s^2}. What is the force the rope is exerting on the truck?

\begin{randomizechoices}
    \choice 8690\,N
    \choice 6489\,N
    \correctchoice 3190\,N
    \choice 810\,N
\end{randomizechoices}

\question
A force of 540\,N is applied to a frictionless sled with a mass 250\,kg. What is the acceleration of the sled?

\begin{randomizechoices}
    \choice \SI{4.5}{m/s^2}
    \correctchoice \SI{2.1}{m/s^2}
    \choice \SI{3.3}{m/s^2}
    \choice \SI{5.5}{m/s^2}
\end{randomizechoices}


\question 
A force of 250\,N is applied to an object that accelerates to \SI{5}{m/s^2}. What is the mass of the object?

\begin{randomizechoices}
    \correctchoice 50\,kg 
    \choice 25\,kg
    \choice 100\,kg
    \choice 15\,kg
\end{randomizechoices}

\question %...Kind of a confusing question. Maybe a ball and bat collide?
A bug and windshield of a fast-moving vehicle collide. Which of the following statements does NOT describe this situation?

\begin{randomizechoices}
    \correctchoice The acceleration of the bug and the vehicle is the same.
    \choice The impulse felt by the bug and the vehicle is the same.
    \choice The force felt by the bug and the vehicle is the same.
    \choice The change in momentum of the bug and the vehicle is the same.
\end{randomizechoices}

\question
Kamaria is learning how to ice skate. She wants her mother to pull her along so that she has an acceleration of \SI{0.80}{m/s^2}. If Kamaria's mass is \SI{27.2}{kg}, with what force does her mother need to pull her? (Neglect any resistance between the ice and Kamaria’s skates.)

\begin{solution}
\phantom{.}

\begin{equation*}
    F_\mathrm{net} = m a = (\SI{27.2}{kg})(\SI{0.80}{m/s^2}) = \boxed{\SI{21.8}{N}}
\end{equation*}
\end{solution}

\question
Two horizontal forces are exerted on a large crate. The first force is 317\,N to the right. The second force is 173\,N to the left. 

\begin{parts}
\part Draw a force diagram for the horizontal forces acting on the crate.

\begin{solution}
\phantom{.}

\begin{center}
    \begin{tikzpicture}[x=0.3,y=0.3]
        \fill (0,0) circle (5pt);
        \draw[very thick,->] (0,0) -- (+317,0) node[right] {317\,N};
        \draw[very thick,->] (0,0) -- (-173,0) node[left] {173\,N};
    \end{tikzpicture}
\end{center}
\end{solution}

\part What is the net force acting on the crate?

\begin{solution}
By the diagram, the net force is

\begin{equation*}
    F_\mathrm{net} = \SI{317}{N} - \SI{173}{N} = \boxed{\SI{144}{N}}
\end{equation*}
\end{solution}

\part The box is initially at rest. Five seconds later, its velocity is 6.5\,m/s to the right. What is the crate's acceleration?

\begin{solution}
\phantom{.}

\begin{equation*}
    a = \frac{\Delta v}{\Delta t} = \frac{\SI{6.5}{m/s} - 0}{\SI{5.0}{s}} = \boxed{\SI{1.3}{m/s^2}}
\end{equation*}
\end{solution}

\part What is the crate's mass?

\begin{solution}
Newton's second law is

\begin{equation*}
    a = \frac{F_\mathrm{net}}{m}
\end{equation*}

Solving this for mass gives

\begin{equation*}
    m = \frac{F_\mathrm{net}}{a} = \frac{\SI{144}{N}}{\frac{1.3}{m/s^2}} = \boxed{\SI{111}{kg}}
\end{equation*}
\end{solution}
\end{parts}

\clearpage

\begin{EnvUplevel}
    \subsection*{Acceleration Station Review (Station 5)}
\end{EnvUplevel}

\question
A cart rolling down an incline for 5 seconds has an acceleration of \SI{4}{m/s^2}.  If the cart has a beginning velocity of \SI{2}{m/s}, what is its final velocity?

\begin{randomizechoices}
    \choice \SI{10}{m/s}
    \correctchoice \SI{22}{m/s}
    \choice \SI{10}{m/s}
    \choice \SI{6}{m/s}
\end{randomizechoices}

\question
A car traveling at a velocity of 30\,m/s encounters an emergency and comes to a complete stop.  How much time will it take for the car to stop if its acceleration is \SI{-4}{m/s^2}?

\begin{randomizechoices}
    \correctchoice \SI{7.5}{s}
    \choice \SI{4.5}{s}
    \choice \SI{-7.5}{s}
    \choice \SI{5.0}{s}
\end{randomizechoices}

\question
The chart below is a record of the instantaneous velocities at the given times for a toy car used in a laboratory investigation.

\begin{center}
    \begin{tabular}{c|c}
        \textbf{Time} (s) & \textbf{Velocity} (m/s) \\ \hline
        0 & $+3$ \\
        1 & $+1$ \\
        2 & $-1$ \\
        3 & $-3$ \\
    \end{tabular}
\end{center}

Which of the following statements is supported by the data?

\begin{randomizechoices}
    \choice From 0 seconds to 1 seconds, the car is speeding up.
    \choice From 1 second to 2 seconds, the car has a constant velocity. 
    \choice From 2 seconds to 3 seconds, the car is slowing down.
    \correctchoice From 0 seconds to 3 seconds, the velocity is changing at a constant rate.
\end{randomizechoices}

\question
Which of the following pair of graphs demonstrates an object moving away from the reference point with a decreasing velocity?

\begin{center}
    \begin{tikzpicture}
        \begin{axis}[height=4cm,
            width=4cm,
            ymin=0,ymax=5,
            xmin=0,xmax=3,
            ticks=none,
            axis lines=left,
            ylabel={$x$},
            y label style={rotate=-90},
            xlabel={$t$},
            title=Graph 1,
        ]
            \addplot[domain=0:2,very thick,black] {4*x - x^2};
        \end{axis}
    \end{tikzpicture}
    \hspace{2em}
    \begin{tikzpicture}
        \begin{axis}[height=4cm,
            width=4cm,
            ymin=0,ymax=5,
            xmin=0,xmax=3,
            ticks=none,
            axis lines=left,
            ylabel={$x$},
            y label style={rotate=-90},
            xlabel={$t$},
            title={Graph 2}
        ]
            \addplot[domain=0:2,very thick,black] {(x-2)^2};
        \end{axis}
    \end{tikzpicture}
    \hspace{2em}
\begin{tikzpicture}
    \begin{axis}[height=4cm,
        width=4cm,
        ymin=-3,ymax=3,
        xmin=0,xmax=3,
        ticks=none,
        axis y line=left,
        axis x line=center,
        ylabel={$v$},
        y label style={rotate=-90},
        xlabel={$t$},
        clip=false,
        title={Graph 3}
    ]
        \addplot[domain=0:2.5,very thick,black] {-x};
        \draw (0,-3) node[below,white] {$t$}; %...FOR ALIGNMNET
    \end{axis}
\end{tikzpicture}
\hspace{2em}
\begin{tikzpicture}
    \begin{axis}[height=4cm,
        width=4cm,
        ymin=-3,ymax=3,
        xmin=0,xmax=3,
        ticks=none,
        axis y line=left,
        axis x line=center,
        ylabel={$v$},
        y label style={rotate=-90},
        xlabel={$t$},
        clip=false,
        title={Graph 4}
    ]
        \addplot[domain=0:2.5,very thick,black] {-x+2.5};
        \draw (0,-3) node[below,white] {$t$}; %...FOR ALIGNMNET
    \end{axis}
\end{tikzpicture}
\end{center}

\begin{randomizechoices}
    \correctchoice Graphs 1 and 4
    \choice Graphs 1 and 3
    \choice Graphs 2 and 4
    \choice Graphs 2 and 3
\end{randomizechoices}

\question
Consider the graph below.

\begin{center}
\begin{tikzpicture}
    \begin{axis}[height=4cm,width=6cm,
        axis lines=left,
        ylabel={Velocity (m/s)},
        xlabel={Time (s)},
        ymin=0,ymax=4,
        xmin=0,xmax=6,
        ytick={0,1,...,4},
        xtick={0,1,...,6},
        grid=both,
    ]
        \addplot[very thick,black,domain=0:5] {3};
    \end{axis}
\end{tikzpicture}
\end{center}

\begin{parts}
\part What is the total displacement of the car between 0 second and 5 seconds?
\begin{solution}
    15 meters
\end{solution}

\part What is the acceleration of the cart?

\begin{solution}
    \SI{0}{m/s^2}
\end{solution}
\end{parts}

\question
The velocity vs. time data table below represents the motion of a 2\,kg object.

\begin{center}
    \begin{tabular}{c|c}
        \textbf{Time} (s) & \textbf{Velocity} (m/s) \\ \hline
        0 & 0 \\ 
        1 & 1 \\
        2 & 2 \\
        3 & 3 \\
        4 & 4 \\
        5 & 5 \\
        6 & 6 \\
    \end{tabular}
\end{center}

\begin{parts}
\part Are the forces on the object balanced or unbalanced?

\begin{solution}
    Unbalanced
\end{solution}

\part What is the acceleration?

\begin{solution}
    \SI{1}{m/s^2}
\end{solution}
\end{parts}

\question
A car has an oil drip. As the car moves in the positive direction, it drips oil at a regular rate, leaving a trail of spots on the road. The oil drop pattern is shown in the figure below.

\begin{center}
\begin{tikzpicture}[scale=0.8]
    \draw[domain=-10:0,mark=*,only marks, samples=8,mark size=3pt] plot({0.1*\x^2},0);
    % \draw[thick, <-] (-5,0.5) -- ++(-1,0) node[above,pos=0.5] {$\vec{v}$};
    \node[above=3pt] at (-10,0) {Start};
    \node[above=3pt] at (0,0) {Finish};
\end{tikzpicture}
\end{center}

\begin{randomizechoices}
    \choice The car is slowing down with a positive acceleration.
    \choice The car is speeding up with a positive acceleration.
    \choice The car is speeding up with a negative acceleration.
    \correctchoice The car is slowing down with a negative acceleration.
\end{randomizechoices}

\question
The velocity vs. time graph represents the motion of a 4\,kg cart along a straight line.

\begin{center}
\begin{tikzpicture}
    \begin{axis}[height=5cm,width=7cm,
        axis lines=left,
        ylabel={Velocity (m/s)},
        xlabel={Time (s)},
        ymin=0,ymax=7,
        xmin=0,xmax=3,
        grid=both,
        ytick={0,1,...,7},
        xtick={0,1,...,3},
        minor x tick num=1,
    ]
        \addplot[very thick,black] coordinates{(0,6)(1,6)(1.25,3)(3,3)};
    \end{axis}
\end{tikzpicture}
\end{center}

which of the following best describes the motion of the cart?

\begin{randomizechoices}
    \choice At Rest, Speeding up, At Rest
    \choice Constant Acceleration, Positive Acceleration, At rest
    \correctchoice Constant Velocity, Slowing down, Constant Velocity
    \choice Positive Acceleration
\end{randomizechoices}

\question
Acceleration is a \fillin[vector][3cm] quantity.  This means it has magnitude and direction.

\begin{randomizechoices}
    \choice scalar
    \choice reference point
    \correctchoice vector
    \choice variable
\end{randomizechoices}

\question
The driver of a car traveling at 30\,m/s slams on the brakes so that the car undergoes a constant acceleration, skidding to a complete stop in 4.5\,s. What is the average acceleration of the car during braking?

\begin{randomizechoices}
    \choice \SI{6.67}{m/s^2}
    \correctchoice \SI{-6.67}{m/s^2}
    \choice \SI{135}{m/s^2}
    \choice \SI{-0.15}{m/s^2}
\end{randomizechoices}

\question
How long will it take a scooter accelerating at \SI{0.400}{m/s^2} to go from rest to a speed of 4.00\,m/s?

\begin{randomizechoices}
    \choice \SI{1.6}{s}
    \choice \SI{0.1}{s}
    \correctchoice \SI{10}{s}
    \choice \SI{4.4}{s}
\end{randomizechoices}

\question
A sudden gust of wind increases the velocity of a sailboat relative to the water surface from 3.0\,m/s to 5.5\,m/s over a period of 30.0\,s. What is the average acceleration of the sailboat?

\begin{randomizechoices}
    \correctchoice \SI{0.083}{m/s^2}
    \choice \SI{0.28}{m/s^2}
    \choice \SI{2.5}{m/s^2}
    \choice \SI{-0.083}{m/s^2}
\end{randomizechoices}

\begin{EnvUplevel}
    \subsection*{Net Force Exit Ticket}
\end{EnvUplevel}

\question
If a net force of \SI{-5}{N} is acting on a 10\,kg object, what is the acceleration of the object?

\begin{randomizechoices}
    \correctchoice \SI{-0.5}{m/s^2}
    \choice \SI{-2}{m/s^2}
    \choice \SI{-5}{m/s^2}
    \choice \SI{-50}{m/s^2}
\end{randomizechoices}

\question
The object depicted below is accelerating at a rate of \SI{2}{m/s^2}. Determine the mass of the object.

\begin{center}
\begin{tikzpicture}
    \draw[fill=cyan!20] (0,0) rectangle (1,1);
    \draw[thick,->,xshift=1cm,yshift=0.5cm] (0,0) -- (3,0) node[right] {15\,N};
    \draw[thick,->,yshift=0.5cm] (0,0) -- (-1,0) node[left] {5\,N};
\end{tikzpicture}
\end{center}

\begin{randomizechoices}
    \choice 2\,kg
    \correctchoice 5\,kg
    \choice 10\,kg
    \choice 20\,kg
\end{randomizechoices}

\question
Three confused sled dogs are trying to pull a sled across the Alaskan snow. Alutia pulls east with a force of 35\,N, Seward also pulls east but with a force of 42\,N, and big Kodiak pulls west with a force of 53\,N. 

\begin{parts}
\part What is the net force on the sled?

\begin{solution}
Net force is

\begin{equation*}
    F_\mathrm{net} = \SI{35}{N} + \SI{42}{N} - \SI{53}{N} = \boxed{\SI{24}{N}}
\end{equation*}    

That is, \SI{24}{N} to the east.
\end{solution}

\part Explain how Newton’s second law accurately predicts the sled’s change in motion.

\begin{solution}
    By Newton's second law, we would conclude that

\begin{equation*}
    F_\mathrm{net} = ma = \SI{24}{N}
\end{equation*}

Thus, if we know the mass of the sled, we can calculate its acceleration.
\end{solution}
\end{parts}

\question
The acceleration of a 16.5\,kg object decreases from \SI{5.0}{m/s^2} to \SI{3.0}{m/s^2}.

\begin{parts}
\part What was the initial net force?

\begin{solution}
Initial net force is

\begin{equation*}
    F_\mathrm{net,i} = ma_i = (\SI{16.5}{kg})(\SI{5.0}{m/s^2}) = \boxed{\SI{82.5}{N}}
\end{equation*}
\end{solution}

\part What was the final net force?

\begin{solution}
Final net force is

\begin{equation*}
    F_\mathrm{net,f} = ma_f = (\SI{16.5}{kg})(\SI{3.0}{m/s^2}) = \boxed{\SI{49.5}{N}}
\end{equation*}
\end{solution}

\part What is the change of the net force on the object?

\begin{solution}
The change in net force is

\begin{equation*}
    \Delta F_\mathrm{net} = F_\mathrm{net,f} - F_\mathrm{net,i} = \boxed{\SI{-33}{N}}
\end{equation*}
\end{solution}
\end{parts}

\question
In order for an object to be in equilibrium, must its velocity or acceleration be constant? Explain.

\begin{solution}
An object is in equilibrium when all forces on it are balanced; that is, when the net force is zero. By Newton's second law, $F_\mathrm{net} = ma$. So, at equilibrium, the object's acceleration must be zero. Therefore, while the object's velocity can be constant or zero, the acceleration must be zero.
\end{solution}


\end{questions}
\end{document}